\documentclass{amsbook}
\usepackage{mathtools}

\newcommand{\setsm}[1]{\left\{#1\right\}}
\newcommand{\given}{\mathrel |}

\newcommand{\wffs}{\mathcal W}

\newcommand{\infers}{\mathrel\vdash}
\newcommand{\theorem}{\mathord\vdash\medspace}

\newcommand{\then}{\mathrel\rightarrow}
\newcommand{\conj}{\mathrel\&}
\newcommand{\eqv}{\mathrel\leftrightarrow}
\newcommand{\disj}{\mathrel\vee}

\theoremstyle{definition}
\newenvironment{hackthm}[2][section]{
    \newtheorem{new}{#2}[#1]
    \begin{new}
}{\end{new}}

\newtheorem{axm}{Axiom}[section]
\newtheorem{frule}{Rule}[section]
\newtheorem{op}{Operator}[section]

\newtheorem{thm}{Theorem}[section]
\newtheorem{lmm}{Lemma}[section]
\newtheorem{dfn}{Definition}[section]

\begin{document}
\title{Metamathema}
\author{Olesj Bilous}
\maketitle

\chapter{Propositional logic}

A proposition $P$ signifies a qualitatively nondescript yet uniquely identifiable fact $P^*$ that is either the case or not.

We may consider the set $\mathcal W$ of all propositions.

\begin{dfn}
    A logical operator $o$ may act on any enumeration $\mathcal E$ of a fixed number $n \in \mathbb N$ of propositions $\mathcal E \subseteq \mathcal W$ to compose a new proposition $o(\mathcal E) = P \in \mathcal W$ where $P \notin \mathcal E$.

    The fixed number $n$ is said to be the order of the operator.

    Any proposition in the enumeration $\mathcal E_i \in \mathcal E$ is said to be in the scope of the operator at position $i$.
\end{dfn}

Consider $P_{i+1} = o((\mathcal E \setminus \setsm {\mathcal E_1}) \cup \setsm {P_i})$ where $P_0 = \mathcal E_1$. Taking $\mathcal D_i = \bigcup_{j=0}^i \setsm {P_j}$ we see that $\mathcal D_i \subset \mathcal D_{i+1} \subseteq \wffs$ so $|\mathcal D_i| = i < |\mathcal W|$ for all $i \in \mathbb N$ hence $|\mathcal W| \notin \mathbb N$.

Evidently there is no bound to the propositions that we may enumerate: if $|\mathcal E| \in \mathbb N$ then $\mathcal E \subset \mathcal W$.

\begin{op}[Material implication]
    For any $P, Q \in \wffs$:
    $$
        P \rightarrow Q \in \wffs.
    $$
    If $(P \rightarrow Q)^*$ is the case, then if $P^*$ (the implicans) is the case, then so is $Q^*$ (the implicandum). The implicans is said to be a sufficient condition for the implicandum.
\end{op}

\begin{op}[Negation]
    For any $P \in \wffs$:
    $$
        \neg P \in \wffs.
    $$
    If $(\neg P)^*$ is the case then $P^*$ is not the case.
\end{op}

Clearly propositions acted on by logical operators may themselves be composed of propositions. This may introduce ambiguity in the notation.

\begin{dfn}
    Parentheses $($, $)$ indicate the scope of a logical operator in a composite proposition.
\end{dfn}

For instance $\neg R$ for $R = P \then Q$ may be written $\neg(P \then Q)$.

Composite propositions may be distinguished from those that contain no logical operators.

\begin{dfn}
    When a proposition $a$ is atomic there is no logical operator $o$ nor any enumeration of propositions $\mathcal E \subset \wffs$ such that $a = o(\mathcal E)$.
\end{dfn}

Atomic propositions are uniquely enumerable without bound.

In the alphabet formed by atomic propositions, logical operators and parentheses not every expression composed of an enumeration of symbols forms a proposition that belongs to the logic. For instance, $\then a$ and $)a \neg$ are ill formed. Therefore $\mathcal W$ is known as the set of well formed formulas.

For purposes of investigation we may introduce an atom symbol $a$ and a succession symbol $'$ such that an exhaustive enumeration $\mathcal E$ of atomic propositions may be denoted $\mathcal E_0 = a$ and $\mathcal E_{i+1} = (\mathcal E_i)'$ for all $i \in \mathbb N$. The alphabet $a, ', \then, \neg, (, )$ permits every expression to be encoded as a natural number in base $6$. Clearly the well formed formulas are enumerable $|\wffs| \leq |\mathbb N|$.

Should we extend the alphabet with a propositional variable symbol $p$ then any expression containing this symbol is evidently ill formed. Nonetheless we may combine it with the succession symbol $'$ to denote instances of propositional variables in a metatheoretical expression.

\begin{dfn}
    Uniform substitution $\varphi(p / P)$ of a propositional variable $p$ with a proposition $P \in \wffs$ substitutes every instance of the variable in the expression $\varphi$ with an instance of the proposition.
\end{dfn}

\begin{dfn}
    A formula $\varphi$ is a metatheoretical expression that becomes a well formed formula $F \in \wffs$ once every propositional variable $p \in \varphi$ has been substituted uniformly with some well formed formula $P \in \wffs$.

    The resulting proposition is known as an instance of the formula $F = \varphi(\mathcal E)$ where $\mathcal E \subset \wffs$ enumerates the substituting propositions in order of substitution.
\end{dfn}

Note that only a formula with no instance of any propositional variable is well formed.

We have used formulas before to define logical operators.
Now we can use them to express the logical structure shared by all instances of a formula.

The set of all formulas shall be denoted $\Phi$.

Formulas in an enumeration $\Lambda \subseteq \Phi$ may contain instances of the same propositional variable.
\begin{dfn}
    An instance $\Lambda(\mathcal E)$ of an enumeration of formulas follows from substitution of all propositional variables uniformly over the enumeration $\Lambda(\mathcal E)_i = \Lambda_i(\mathcal E)$.
\end{dfn}

\begin{dfn}
    An enumeration of propositions $\mathcal E \subseteq \wffs$ may allow us to syntactically derive $\mathcal E \infers P$ a proposition $P$.

    In that case the propositions of $\mathcal E$ are known as premises. The conclusion $P$ is said to be a syntactic consequence.
\end{dfn}


Syntactic derivation is transitive such that if $\mathcal E \infers P$ for all $P \in \mathcal D$ and $\mathcal D \infers Q$ then $\mathcal E \infers Q$.

\begin{dfn}
    The deductive closure $\partial \mathcal E$ of an enumeration of propositions $\mathcal E \subseteq \wffs$ contains all syntactic consequences $\mathcal E \infers P$.
\end{dfn}

\begin{dfn}
    A rule of inference $\Lambda \infers \kappa$ relates a finite enumeration of formulas $(\Lambda \cup \setsm \kappa) \subset \Phi$ such that $\Lambda(\mathcal E) \infers \kappa(\mathcal E)$ for every instance $(\Lambda \cup \setsm \kappa)(\mathcal E)$ of the enumeration.
\end{dfn}

\begin{frule}[Modus ponens]
    $$\begin{aligned}p \rightarrow q, \\ p\end{aligned} \infers q$$
\end{frule}

\section{Inferential approach}

The following special rules of inference allow us to infer over the possibility of inference itself. Consider the extension $\mathcal E \cup\setsm P$ of some premises $\mathcal E$ with a proposition $P$ known as the hypothesis.

\begin{frule}[Conditional proof]
    If $\Lambda \cup\setsm p \infers q$ then
    $$\Lambda \infers p \then q.$$
    for all $\Lambda \subset \Phi$.
\end{frule}

Hence the subject of hypothesis is sufficient condition for what can be inferred from the extension.

This entails a familiar property.

\begin{thm}[Linearity of inference]
    If $\Lambda \infers p$ and $\Lambda \cup \setsm p \infers q$ then $\Lambda \infers q$ for all $\Lambda \subset \Phi$.
    \begin{proof}
        From $\Lambda \infers p \then q$ and transitivity of syntactic derivation.
    \end{proof}
\end{thm}

Some expected properties of the implication can now follow.

\begin{thm}[Transitivity of implication]
    $$\begin{aligned}
            p \then q, \\
            q \then r
        \end{aligned} \infers p \then r$$
    \begin{proof}
        From $p \then q, q \then r, p \infers r$.
    \end{proof}
\end{thm}

\begin{thm}[Distributivity of implication]
    $$p \then (q \then r) \infers (p \then q) \then (p \then r)$$
\end{thm}

The following rule seems all too evident.

\begin{frule}[Reflection]
    $$p \infers p$$
\end{frule}

It allows us to prove the following theorems.

\begin{thm}[Reflexivity of implication]
    $$\theorem p \then p$$
\end{thm}

\begin{thm}[Materiality of implication]
    $$p \infers q \then p$$
    \begin{proof}
        $p, q \infers p$.
    \end{proof}
\end{thm}

The preceding rules and theorems have applied exclusively to the implication. The following rules introduce the syntactic role of the negation.

\begin{frule}[Contraposition]
    $$p \then q \infers \neg q \then \neg p$$
\end{frule}

\begin{frule}[Introduction of double negation]
    $$p \infers \neg\neg p$$
\end{frule}

This immediately establishes some interesting theorems.

\begin{thm}[Commutativity of conjunction]
    $$p \then \neg q \infers q \then \neg p$$
    \begin{proof}
        $q \then \neg\neg q$ is transitive with $\neg\neg q \then \neg p$.
    \end{proof}
\end{thm}

\begin{thm}[Introduction of conjuction]
    $$p, q \infers \neg(p \then \neg q)$$
    \begin{proof}
        $p \infers (p \then \neg q) \then \neg q$.
    \end{proof}
\end{thm}

\begin{thm}[Reductio ad absurdum]
    $$\begin{aligned}
            p \then q, \\ p \then \neg q
        \end{aligned} \infers \neg p$$
    \begin{proof}
        $p \then q, p \then \neg q \infers q \then \neg p, p \then \neg p$ such that $p, p \then q \infers (p \then \neg q) \then \neg p, p \then \neg(p \then \neg q)$ and $p \then q \infers (p \then \neg q) \then \neg p$.
    \end{proof}
\end{thm}

The final rule completes the law of double negation.

\begin{frule}[Elimination of double negation]
    $${\neg\neg p} \infers p$$
\end{frule}

This rule has been controversial since it allows the following.

\begin{thm}[Proof by contradiction]
    $$\neg p \then q, \neg p \then \neg q \infers p$$
\end{thm}

\begin{thm}[Ex falso quod libet]
    $$\begin{aligned}
            p, \\ \neg p
        \end{aligned}\infers q$$
    \begin{proof}
        Evidently $p, \neg p \infers \neg q \then p, \neg q \then \neg p$.
    \end{proof}
\end{thm}

It also completes the laws of contraposition and conjunction.

\begin{lmm}[Contraposition with right elimination]
    $$\neg p \then q \infers \neg q \then p$$
\end{lmm}

\begin{thm}[Converse contraposition]
    $$\neg p \then \neg q \infers p \then q$$
\end{thm}

\begin{thm}[Elimination of conjunction]
    $$\neg(p \then \neg q) \infers p, q$$
    \begin{proof}
        $\neg q, \neg p \infers p \then \neg q$.
    \end{proof}
\end{thm}


\begin{thm}[Peirce's law]
    $$(p \then q) \then p \infers p$$
    \begin{proof}
        We know that $\theorem \neg p \then (\neg q \then \neg p)$ so $\theorem \neg p \then (p \then q)$. However, in that case $\neg p \then \neg(p \then q) \infers p$.
    \end{proof}
\end{thm}

Consider now $\theorem \neg p \then (p \then \neg p)$.

In that case $\theorem \neg(p \then \neg p) \then \neg\neg p$


\section{Alternative operators}


\begin{op}[Conjuction]
    We define $P \conj Q$ as $\neg(P \then \neg Q)$.
\end{op}

Double negation introduction shows $\theorem \neg(p \conj \neg p)$ by $\theorem \neg\neg(p \then \neg\neg p)$.

Furthermore $p \then q$ can be expressed as $\neg(p \conj \neg q)$.

By contraposition with left elimination we have commutativity $p \conj q \infers q \conj p$.


Clearly introduction of conjuction translates to $p, q \infers p \conj q$.

Since $\theorem p \then (q \then \neg(p \then \neg q))$, a theorem deduced from modus ponens, we have $\theorem \neg(p \conj (q \conj \neg(p \conj q)))$.

From $\theorem p \then (q \then p)$ we have $\neg(p \conj (q \conj \neg p)) $.

Since $\theorem (p \then (q \then r)) \then ((p \then q) \then (q \then r))$ we have $\neg((\neg r \conj q) \conj p) \infers \neg((p \conj q) \conj \neg(p \conj r))$.

We need double negation elimination to introduce an inferential analogue of modus ponens $\neg(p \conj \neg q), p \infers q$ as well as to complete the law of conjuction $p \conj q \infers p, q$.

Reductio ad absurdum may be expressed $\neg(p \conj q), \neg(p \conj \neg q) \infers \neg p$.

\begin{op}[Equivalence]
    We define $P \eqv Q$ as $(P \then Q) \conj (Q \then P)$.
\end{op}

\begin{op}[Disjunction]
    We define $P \disj Q$ as $\neg P \then Q$.
\end{op}

We have $p \disj q, \neg p \infers q$ modus tollens.

Distributivity becomes $p \disj (q \disj r) \infers (p \disj r) \disj \neg(p \disj \neg q)$.

Double negation elimination translates to $\theorem p \disj \neg p$.

Along with contraposition this means $p \disj q \infers q \disj p$.

The verum ex quod libet may then be rewritten as $p \infers p \disj q$.

Interestingly $\neg p \then \neg(q \then \neg r), \neg p \infers q, r$ hence $p \disj (q \conj r) \infers (p \disj q) \conj (p \disj r)$.

Peirce's law becomes $p \vee (p \conj q) \infers p$.


\newpage


This justifies the following definition.

The following result is a formulation of the law of the excluded middle in terms of the implication.

Clearly $(\neg p \then p) \then \neg p \infers \neg p$.

We see that $\theorem p \then (\neg p \then p)$ and $\theorem (\neg p \then p) \then \neg\neg p$

Since $(\neg\neg p \then p) \then p \infers (\neg\neg p \then p) \then \neg\neg p, \neg\neg p$.

Furthermore $\neg\neg p \infers \neg(p \then \neg p)$ since $\theorem (p \then \neg p)\then \neg p$.

This becomes more clear once we consider $\theorem (p \then \neg p) \then \neg p$.

In that case $\theorem \neg p \then \neg p$ means $\theorem (p \then \neg p)\then $.

Now $(\neg\neg p \then p) \then \neg p \infers p \then \neg p$
since $\theorem p \then (\neg\neg p \then p)$.

Hence $\theorem ((\neg\neg p \then p) \then \neg p) \then \neg p$.

\begin{dfn}
    We may define the disjunction $P \vee Q$ as $\neg P \then Q$ and $\neg Q \then P$.
\end{dfn}

The law of the excluded middle is then defined as $p \vee \neg p$. Evidently this is equivalent to reflexivity and introduction of double negation.

We may rewrite Peirce's law as $p \vee (p \and \neg p) \infers p$ for the instance $(p \then \neg\neg p) \then p \infers p$.

\newpage

Including this assumption with the preceding rules and operators suffices to characterise the entire logic. This is known as the inferential approach.


We may now show some interesting theorems that follow from the inferential approach. However, we may also characterise the logic by means of these theorems as premises while dropping the assumption of reflexivity as well as every rule except modus ponens. In that case these characterising theorems become axioms from which reflexivity and the omitted rules may be inferred.

Each of the characterising theorems of the inferential approach has a characteristic logical structure in which any of the composing propositions may be substituted for any other proposition to yield another characterising theorem. Conversely, the axioms may be written in a characteristic general form where substitution of a composing proposition yields another axiom.

\begin{axm}[Materiality of implication]
    $$P \then (Q \then P)$$
\end{axm}

This shows that $P \then (P \then P)$ is an axiom so $P \infers P$ for any $P, Q \in \mathcal W$. Hence the logic remains reflexive under the axiomatic approach. Evidently this implies that any axiom is also a theorem, which is necessary for the equivalence of the inferential and axiomatic approach.

Next we see that $P \then (Q \then R), P \then Q, P \infers R$.

\begin{axm}[Distributivity of implication]
    $$(P \then (Q \then R)) \then ((P \then Q) \then (P \then R))$$
\end{axm}

\end{document}
