\documentclass{amsbook}
\usepackage{mathtools}
\usepackage{amssymb}
\usepackage{enumitem}

\newcommand{\setsm}[1]{\left\{#1\right\}}
\newcommand{\given}{\mathrel |}

\newcommand{\wffs}{\mathcal W}

\newcommand{\infers}{\mathrel\vdash}
\newcommand{\theorem}{\mathord\vdash\medspace}
\newcommand{\univ}[1]{\mathord\forall#1\;}
\newcommand{\exis}[1]{\mathord\exists#1\;}

\newcommand{\valids}{\mathrel\vDash}


\newcommand{\then}{\mathrel\rightarrow}
\newcommand{\conj}{\mathrel\&}
\newcommand{\eqv}{\mathrel\leftrightarrow}
\newcommand{\disj}{\mathrel\vee}

\theoremstyle{definition}
\newenvironment{hackthm}[2][section]{
    \newtheorem{new}{#2}[#1]
    \begin{new}
}{\end{new}}

\newtheorem{axm}{Axiom}[chapter]
\newtheorem{prop}{Property}[section]
\newtheorem{subprop}{Property}[subsection]
\newtheorem{frule}{Rule}[chapter]
\newtheorem{subrule}{Rule}[subsection]
\newtheorem{op}{Operator}[chapter]

\newtheorem{quant}{Quantifier}[chapter]

\newtheorem{thm}{Theorem}[section]
\newtheorem{subthm}{Theorem}[subsection]
\newtheorem{lmm}{Lemma}[section]
\newtheorem{crl}{Corollary}[section]
\newtheorem{dfn}{Definition}[section]


\begin{document}


\title{Metamathema}
\author{Olesj Bilous}
\maketitle

\chapter{Propositional logic}

A proposition $P$ signifies a qualitatively nondescript yet uniquely identifiable fact $P^*$ that is either the case or not.

\section{Syntax}

\begin{dfn}
    A logical operator $o$ may act on any enumeration $\mathcal E$ of a fixed number $n \in \mathbb N$ of propositions $\mathcal E \subseteq \mathcal W$ to compose a new proposition $o(\mathcal E) = P \in \mathcal W$ where $P \notin \mathcal E$.

    The fixed number $n$ is said to be the order of the operator.

    Any proposition in the enumeration $\mathcal E_i \in \mathcal E$ is said to be in the scope of the operator at position $i$.
\end{dfn}

\begin{thm}
    There is no bound to the propositions that we may enumerate: if $|\mathcal E| \in \mathbb N$ then $\mathcal E \subset \mathcal W$.
    \begin{proof}
        Consider $P_{i+1} = o((\mathcal E \setminus \setsm {\mathcal E_1}) \cup \setsm {P_i})$ where $P_0 = \mathcal E_1$.

        Taking $\mathcal D_i = \bigcup_{j=0}^i \setsm {P_j}$ we see that $\mathcal D_i \subset \mathcal D_{i+1} \subseteq \wffs$ so $|\mathcal D_i| = i < |\mathcal W|$ for all $i \in \mathbb N$ hence $|\mathcal W| \notin \mathbb N$.
    \end{proof}
\end{thm}


\begin{op}[Material implication]
    For any $P, Q \in \wffs$:
    $$
        P \rightarrow Q \in \wffs.
    $$
    If $(P \rightarrow Q)^*$ is the case, then if $P^*$ (the implicans) is the case, then so is $Q^*$ (the implicandum). The implicans is said to be a sufficient condition for the implicandum.
\end{op}

\begin{op}[Negation]
    For any $P \in \wffs$:
    $$
        \neg P \in \wffs.
    $$
    If $(\neg P)^*$ is the case then $P^*$ is not the case.
\end{op}

Clearly propositions acted on by logical operators may themselves be composed of propositions. This may introduce ambiguity in the notation.

\begin{dfn}
    Parentheses $($, $)$ indicate the scope of a logical operator in a composite proposition.
\end{dfn}

For instance $\neg R$ for $R = P \then Q$ may be written $\neg(P \then Q)$.

Composite propositions may be distinguished from those that contain no logical operators.

\begin{dfn}
    When a proposition $a$ is atomic there is no logical operator $o$ nor any enumeration of propositions $\mathcal E \subset \wffs$ such that $a = o(\mathcal E)$.

    An exhaustive enumeration of atomic propositions is possible yet has no bound.
\end{dfn}

In the alphabet formed by atomic propositions, logical operators and parentheses not every expression composed of an enumeration of symbols forms a proposition that belongs to the logic. For instance, $\then a$ and $)a \neg$ are ill formed. Therefore $\mathcal W$ is known as the set of well formed formulas.

For purposes of investigation we may introduce an atom symbol $a$ and a succession symbol $'$ such that an exhaustive enumeration $\mathcal E$ of atomic propositions may be denoted $\mathcal E_0 = a$ and $\mathcal E_{i+1} = (\mathcal E_i)'$ for all $i \in \mathbb N$. The alphabet $a, ', \then, \neg, (, )$ permits every expression to be encoded as a natural number in base $6$.

\begin{thm}
    The well formed formulas are exhaustively enumerable $|\wffs| \leq |\mathbb N|$.
\end{thm}

Should we extend the alphabet with a propositional variable symbol $p$ then any expression containing this symbol is evidently ill formed. Nonetheless we may combine it with the succession symbol $'$ to denote instances of propositional variables in a metatheoretical expression.

\begin{dfn}
    Uniform substitution $\varphi(p / P)$ of a propositional variable $p$ with a proposition $P \in \wffs$ substitutes every instance of the variable in the expression $\varphi$ with an instance of the proposition.
\end{dfn}

\begin{dfn}
    A formula $\varphi$ is a metatheoretical expression that becomes a well formed formula $F \in \wffs$ once every propositional variable $p \in \varphi$ has been substituted uniformly with some well formed formula $P \in \wffs$.

    The resulting proposition is known as an instance of the formula $F = \varphi(\mathcal E)$ where $\mathcal E \subset \wffs$ enumerates the substituting propositions in order of substitution.
\end{dfn}

Note that only a formula with no instance of any propositional variable is well formed.

We have used formulas before to define logical operators.
Now we can use them to express the logical structure shared by all instances of a formula.

The set of all formulas shall be denoted $\Phi$.

Different formulas may contain instances of the same propositional variable.

\begin{dfn}
    An instance $\varLambda(\mathcal E)$ of a set of formulas $\varLambda \in \Phi$ follows from substitution of all propositional variables uniformly over the set $\varLambda(\mathcal E)_i = \varLambda_i(\mathcal E)$.
\end{dfn}

\begin{dfn}
    A set of propositions $\mathcal E \subseteq \wffs$ may allow us to infer $\mathcal E \infers P$ a proposition $P$.

    In that case the propositions of $\mathcal E$ are known as premises whereas $P$ is said to be a syntactic consequence.

    A rule of inference $\Lambda \infers \kappa$ relates a finite set of formulas $(\varLambda \cup \setsm \kappa) \subset \Phi$ such that $\varLambda(\mathcal E) \infers \kappa(\mathcal E)$ for every instance $(\varLambda \cup \setsm \kappa)(\mathcal E)$.

    Every inference is an instance of some rule of inference. This ensures that inference is structural.
\end{dfn}

\begin{dfn}
    The deductive closure $\partial \mathcal E$ of a set of propositions $\mathcal E \subseteq \wffs$ contains all syntactic consequences of the set such that $\mathcal E \infers P$ only if $P \in \partial\mathcal E$.

    The set is deductively closed if $\mathcal E = \partial\mathcal E$.

    The set is consistent if there is no $P \in \partial\mathcal E$ such that also $\neg P \in \partial\mathcal E$.

    The set is trivial if $\partial\mathcal E = \wffs$.

    A set which is consistent or not trivial is maximally so if its extension $\mathcal E \cup \setsm P$ with any $P \notin \mathcal E$ is no longer so.
\end{dfn}

\begin{prop}[Transitivity]
    If $\mathcal E \infers P$ for all $P \in \mathcal D$ and $\mathcal D \infers Q$ then $\mathcal E \infers Q$.
\end{prop}

The following property does not hold in adaptive inference systems.

\begin{prop}[Monotony]
    If $\mathcal E \infers P$ then $\mathcal E \cup \setsm Q \infers P$ for any $Q$.
\end{prop}

\begin{frule}[Modus ponens]
    $$\begin{aligned}p \rightarrow q, \\ p\end{aligned} \infers q$$
\end{frule}

\subsection{Inferential approach}

The following property allows us to infer over the possibility of inference itself. Consider the extension $\mathcal E \cup\setsm P$ of some premises $\mathcal E$ with a proposition $P$ known as the hypothesis.

\begin{subprop}[Deduction]
    If $\mathcal E \cup\setsm P \infers Q$ then
    $$\mathcal E \infers P \then Q.$$
    for all $\mathcal E \subset \wffs$.
\end{subprop}

Hence the subject of hypothesis is sufficient condition for what can be inferred from the extension.

This holds a fortiori for any finite $\varLambda \subset \Phi$ such that $\varLambda \cup \setsm p \infers q$ since it holds for all instances.

Furthermore this entails a familiar property.

\begin{thm}[Linearity of inference]
    If $\varLambda \infers p$ and $\varLambda \cup \setsm p \infers q$ then $\varLambda \infers q$ for all $\varLambda \subset \Phi$.
    \begin{proof}
        From $\varLambda \infers p \then q$ and transitivity of inference.
    \end{proof}
\end{thm}

Some expected properties of the implication can now follow.

\begin{thm}[Transitivity of implication]
    $$\begin{aligned}
            p \then q, \\
            q \then r
        \end{aligned} \infers p \then r$$
    \begin{proof}
        From $p \then q, q \then r, p \infers r$.
    \end{proof}
\end{thm}

\begin{thm}[Distributivity of implication]
    $$p \then (q \then r) \infers (p \then q) \then (p \then r)$$
\end{thm}

The following rule seems all too evident.

\begin{subrule}[Reflection]
    $$p \infers p$$
\end{subrule}

It allows us to prove the following theorems.

\begin{thm}[Reflexivity of implication]
    $$\theorem p \then p$$
\end{thm}

\begin{thm}[Ex quod libet datum]
    $$p \infers q \then p$$
    \begin{proof}
        $p, q \infers p$ by monotony.
    \end{proof}
\end{thm}

The implication thus defined is said to be material.

The preceding rules and theorems have applied exclusively to the implication. The following rule introduces the syntactic role of the negation.

\begin{frule}[Contraposition]
    $$p \then q \infers \neg q \then \neg p$$
\end{frule}

This shows a property of inference.

\begin{thm}[Contrapositivity]
    If $p \infers q$ then $\neg q \infers \neg p$.
\end{thm}

\begin{thm}[Ex absurdum negatio quod libet]
    $$p, \neg p \infers \neg q$$
    \begin{proof}
        $p \infers q \then p$.
    \end{proof}
\end{thm}

\begin{crl}[Ex absurdo falsum]
    For any formula $\tau$ such that $\theorem \tau$:
    $$p, \neg p \infers \neg\tau$$
\end{crl}

\begin{thm}[Ex falso negatio quod libet]
    For any formula $\tau$ such that $\theorem \tau$:
    $$\theorem \neg\tau \then \neg q$$
    \begin{proof}
        $\theorem q \then \tau$, also known as ex quod libet verum.
    \end{proof}
\end{thm}

\begin{frule}[Introduction of double negation]
    $$p \infers \neg\neg p$$
\end{frule}

\begin{lmm}[Contraposition with left elimination]
    $$p \then \neg q \infers q \then \neg p$$
    \begin{proof}
        $q \then \neg\neg q$ is transitive with $\neg\neg q \then \neg p$.
    \end{proof}
\end{lmm}

\begin{thm}[Negatio improbis]
    For any formula $\tau$ such that $\theorem \tau$:
    $$p \then \neg \tau \infers \neg p.$$
\end{thm}

\begin{thm}[Introduction of implicative conjuction]
    $$p, q \infers \neg(p \then \neg q)$$
    \begin{proof}
        We can deduce $p \infers (p \then \neg q) \then \neg q$ from modus ponens.
    \end{proof}
\end{thm}

This has a special case with an interesting consequence.

\begin{lmm}[Negatio reprobis]
    $$p \then \neg p \infers \neg p$$
    \begin{proof}
        $p \infers \neg(p \then \neg p)$.
        Alternatively by ex contradictio falsum.
    \end{proof}
\end{lmm}

\begin{thm}[Reductio ad absurdum]
    $$\begin{aligned}
            p \then q, \\ p \then \neg q
        \end{aligned} \infers \neg p$$
\end{thm}

By contrapositivity we also have the following.

\begin{crl}[Commutativity of implicative conjunction]
    $$\neg(p \then \neg q) \infers \neg(q \then \neg p)$$
\end{crl}

The final rule completes the law of double negation.

\begin{frule}[Elimination of double negation]
    $${\neg\neg p} \infers p$$
\end{frule}

This rule has been controversial since it allows the following.

\begin{thm}[Proof by contradiction]
    $$\neg p \then q, \neg p \then \neg q \infers p$$
\end{thm}

This result is rejected by intuitionists, hence they do not allow double negation to be eliminated.

The following completes the implicative variety of the law of junctive commutativity.

\begin{lmm}[Commutativity of implicative disjunction]
    $$\neg p \then q \infers \neg q \then p$$
\end{lmm}

This leads to double negation elimination considering $\theorem \neg p \then \neg\neg\neg p$.

Nonetheless, the following results may be preserved by intuitionists as independent rules. We complete the law of implicative conjuction and show an interesting lemma.

\begin{crl}[Elimination of implicative conjunction]
    $$\neg(p \then \neg q) \infers p, q$$
    \begin{proof}
        $\neg p \infers p \then \neg q$.
    \end{proof}
\end{crl}

\begin{lmm}[Introduction of implicative disjunction]
    $$p \infers \neg p \then q$$
    \begin{proof}
        $p \infers \neg q \then p$.
    \end{proof}
\end{lmm}

\begin{thm}[Ex absurdo quod libet]
    $$\begin{aligned}
            p, \\ \neg p
        \end{aligned}\infers q$$
\end{thm}

A consistent set is clearly not trivial. Now we can also show the following.

\begin{crl}
    An inconsistent set is trivial.
\end{crl}

Hence a maximally consistent set is maximally not trivial. This is avoided in paraconsistent inferential systems.

For any formula $\tau$ such that $\theorem \tau$:

\begin{thm}[Ex falso quod libet]
    $$\theorem \neg\tau \then q$$
\end{thm}

\begin{crl}[Ex falso absurdum]
    $$\neg\tau \infers p, \neg p$$
\end{crl}

Furthermore the completion of the law of contraposition allows us to conclude with a remarkable result.

\begin{lmm}[Converse contraposition]
    $$\neg p \then \neg q \infers q \then p$$
\end{lmm}

\begin{thm}[Peirce's law]
    $$(p \then q) \then p \infers p$$
    \begin{proof}
        We know that $\theorem \neg p \then (\neg q \then \neg p)$ so $\theorem \neg p \then (p \then q)$. However, in that case $\neg p \then \neg(p \then q) \infers p$.
    \end{proof}
\end{thm}

\begin{crl}[Consequentia mirabilis]
    $$\neg p \then p \infers p$$
    \begin{proof}
        $\theorem (p \then \neg p) \then \neg p$ is transitive with $\neg p \then p$.
    \end{proof}
\end{crl}

By ex absurdum quod libet we can show $\neg\neg p, \neg p \infers p$. Evidently Peirce's law is unacceptable to intuitionists.

The following rule is rejected in its implicative variety for similar reasons.

\begin{thm}[Implicative dilemma]
    $$\neg p \then q, \begin{aligned}
            p \then r \\
            q \then r
        \end{aligned} \infers r$$
    \begin{proof}
        $p \then q, \neg p \then q, \neg q \infers q$.
    \end{proof}
\end{thm}

Since $\theorem p \then p$ this is enough to invite consequentia mirabilis.

\subsection{Alternative operators}

We may disregard the law of double negation and introduce some new operators.

\begin{op}[Conjuction]
    If $(P \conj Q)^*$ is the case, then $P^*$ is the case and $Q^*$ is the case.
\end{op}

\begin{frule}[Law of conjunction]
    $p, q \infers p \conj q$ and $p \conj q \infers p, q$.
\end{frule}

\begin{thm}[Conjunctive syllogism]
    $\neg(p \conj q), p \infers \neg q$.
    \begin{proof}
        $p \infers q \then (p \conj q)$.
    \end{proof}
\end{thm}

\begin{frule}[Law against contradiction]
    $$\theorem \neg(p \conj \neg p)$$
\end{frule}

\begin{lmm}
    $\theorem p \then \neg\neg p$.
\end{lmm}

\begin{lmm}
    $p \conj q \infers \neg(p \then \neg q)$
\end{lmm}

\begin{lmm}
    $p \then q \infers \neg(p \conj \neg q)$.
\end{lmm}

\begin{op}[Disjunction]
    If $(P \disj Q)^*$ is the case, then $P^*$ is the case or $Q^*$ is the case.
\end{op}

\begin{frule}[Disjunctive syllogism]
    $$p \disj q, \neg p \infers q$$
\end{frule}

\begin{lmm}
    $\neg p \disj q \infers p \then q$.
\end{lmm}

\begin{frule}[Introduction of disjunction]
    $$p \infers p \disj q$$
\end{frule}

\begin{frule}[Dilemma]
    $$p \disj q,\begin{aligned}
            p \then r, \\ q \then r
        \end{aligned} \infers r$$
\end{frule}

\begin{frule}[Law of junctive commutativity]
    $p \conj q \infers q \conj p$ and $p \disj q \infers q \disj p$.
\end{frule}

\begin{op}[Equivalence]
    If $(P \eqv Q)^*$ is the case then $P^*$ is the case only if $Q^*$ is the case.
\end{op}

\begin{frule}[Law of equivalence]
    \begin{align*}
        p \then q, q \then p & \infers p \eqv q             \\
        p \eqv q             & \infers p \then q, q \then p
    \end{align*}
\end{frule}

\begin{thm}
    $\theorem (p \eqv q) \eqv ((p \then q) \conj (q \then p))$
\end{thm}

\begin{thm}[De Morgan's laws I]
    $$\theorem \neg(p \disj q) \eqv (\neg p \conj \neg q)$$
    \begin{proof}
        $\theorem p \then (p \disj q)$, $\theorem q \then (p \disj q)$. $\neg p \infers (p \disj q) \then q$.
    \end{proof}
\end{thm}

\begin{thm}[De Morgan's laws IIa]
    $$\neg p \disj \neg q \infers \neg(p \conj q)$$
    \begin{proof}
        $\neg p \disj \neg q, p \conj q \infers q, \neg q$.
    \end{proof}
\end{thm}

\begin{frule}[Law of the excluded middle]
    $$\theorem \neg p \disj p$$
\end{frule}

\begin{crl}
    $\theorem \neg\neg p \then p$.
\end{crl}

This is the first offence against intuitionism in this approach.

A weaker variety of the offending law suffices to complete De Morgan's laws.

\begin{lmm}
    $\theorem \neg p \disj \neg\neg p$
\end{lmm}

\begin{thm}[De Morgan's laws IIb]
    $$\neg(p \conj q) \infers \neg p \disj \neg q$$
    \begin{proof}
        $\neg(p \conj q) \infers \neg\neg p \then (\neg p \disj \neg q)$ from $p \then \neg q \infers \neg\neg p \then \neg q$.
    \end{proof}
\end{thm}

\begin{thm}
    \begin{align*}
         & \theorem & (p \conj q) &  & \eqv &  & \neg & (p \then \neg q) & \\
         & \theorem & (p \disj q) &  & \eqv &  &      & (\neg p \then q) &
    \end{align*}
    \begin{proof}
        From elimination of implicative conjuction and from $\neg p \then q \infers \neg p \then (p \disj q)$ by dilemma over exclusion of the middle.
    \end{proof}
\end{thm}

\begin{crl}
    \begin{align*}
         & \theorem & (p \then q) &  & \eqv &  & \neg & (p \conj \neg q) & \\
         & \theorem & (p \then q) &  & \eqv &  &      & (\neg p \disj q) &
    \end{align*}
\end{crl}

\subsection{Axiomatic approach}

We may disregard all properties and rules from the inferential approach. Instead we consider a special set of formulas $\Xi \subset \Phi$ known as axioms. We may also consider the set of all instances of axioms $\mathcal X \subset \wffs$.

\begin{prop}
    If $\mathcal E \subset \setsm {Q, Q \then P} \subset (\mathcal E \cup \mathcal X)$ then $\mathcal E \infers P$.
\end{prop}

This property is independent from standard modus ponens yet extends it. It is also independent from monotony.

\begin{axm}
    $p \then (q \then p)$
\end{axm}

\begin{axm}
    $(p \then (q \then r)) \then ((p \then q) \then (p \then r))$
\end{axm}

\begin{lmm}
    $\theorem p \then p$
    \begin{proof}
        Taking $\alpha = p \then ((q \then p) \then p)$ and $\beta = p \then (q \then p)$ we have $\alpha \then (\beta \then (p \then p)), \alpha, \beta \in \Xi$.
    \end{proof}
\end{lmm}

\begin{crl}
    $p \infers p$
\end{crl}

\begin{thm}[Deduction]
    If $\varLambda \cup \setsm p \infers q$ then $\varLambda \infers p \then q$.
    \begin{proof}
        We have $\varLambda \infers p \then \alpha$ for any $\alpha \in \varLambda \cup \setsm p$.

        Hence for any $\alpha \then \beta \in \varLambda \cup \setsm p$ we have $\varLambda \infers p \then (\alpha \then \beta)$.

        Evidently $\varLambda \infers p \then \beta$ permits continued deduction until we show $\varLambda \infers p \then \gamma$ for any $\gamma \in \partial(\varLambda \cup \setsm p)$.
    \end{proof}
\end{thm}

\begin{axm}
    $(\neg p \then \neg q) \then (q \then p)$
\end{axm}

\begin{thm}[Double negation elimination]
    $\theorem \neg\neg p \then p$
    \begin{proof}
        $\neg\neg p \infers \neg\neg(\neg\neg p) \then \neg\neg p$.
    \end{proof}
\end{thm}

\begin{thm}[Double negation introduction]
    $\theorem p \then \neg\neg p$
    \begin{proof}
        $\theorem \neg\neg(\neg p) \then \neg p$.
    \end{proof}
\end{thm}

Evidently the axiomatic approach allows precisely the same syntactic derivations as the inferential.


\section{Semantics}

The expressive range of $P$ is restricted to a binary distinction $P^* \in \setsm{0,1}$.

We say that $P^*$ is the case only if $P^* = 1$. Only in that case is $P$ true.

\begin{dfn}
    A model is a valuation map $v: \wffs \longrightarrow \setsm{0,1}$ that assigns a truth value $v(P) \in \setsm{0,1}$ to each proposition $P \in \wffs$.

    Support for $(\mathcal E, P)$ by a model $v$ means that if all $E \in \mathcal E$ assign to $v(E) = 1$, then $v(P) = 1$. Furthermore any $v$ is:
    \begin{description}[
            labelindent=\parindent,
            before={
                    \renewcommand\makelabel[1]{(##1).}
                }
        ]
        \item [inferential] If $\mathcal E \infers P$ then $v$ supports $(\mathcal E, P)$.
        \item [regular with respect to negation] If $v(P) = v(Q)$ then $v(\neg P) = v(\neg Q)$.
        \item [not trivial] There is at least one proposition $P$ such that $v(P) = 0$.
    \end{description}
    We may consider the set of all models $\mathbf M$.
\end{dfn}

The positive domain $v^{-1}(1) \subseteq \wffs$ of a valuation map $v$ contains all propositions $P$ for which $v(P) = 1$. Clearly the model is determined by the positive domain alone. A model $v$ supports $(\mathcal E, P)$ only if $P \in v^{-1}(1)$ if $\mathcal E \subseteq v^{-1}(1)$. We can see that a valuation map is inferential only if the positive domain is deductively closed.

\begin{dfn}
    A proposition is a semantic consequence $\mathcal E \valids P$ of a set of premises only if all models $v \in \mathbf M$ support $(\mathcal E, P)$.

    $\mathcal E$ is said to validate $P$.
\end{dfn}

\begin{lmm}
    If $\mathcal E \infers P$ then $\mathcal E \valids P$.

    The inferential system is semantically sound. The semantics are adequate for the inferential system.
\end{lmm}

\begin{prop}
    The following properties characterise the negation.
    \begin{description}[
            labelindent=\parindent,
            before={
                    \renewcommand\makelabel[1]{(##1).}
                }
        ]
        \item[consistency] $v(\neg P) = 0$ if $v(P) = 1$
        \item[completeness] $v(\neg P) = 1$ if $v(P) = 0$
    \end{description}

    \begin{proof}
        A model must be consistent since it may not be trivial.

        Consider now that $\theorem T$ where $T = a \then \neg\neg a$. Clearly $v(T) = 1$ so $v(\neg T) = 0$ by consistency. However, $\theorem \neg\neg T$ from which $v(\neg P) = 1$ if $v(P) = 0$ by negational regularity.
    \end{proof}
\end{prop}

A consistent map $v$ is not trivial. If it is also complete, it is regular with respect to negation. Hence, an inferential valuation map is regular with respect to negation and not trivial only if it is consistent and complete.

From paraconsistent inferential systems we may derive inferential maps which are not trivial yet inconsistent. Moreover, regularity with respect to negation implies that a single contradiction $v(P) = 1 = v(\neg P)$ of some $P$ supports the contradiction $v(\neg Q) = 1$ of each $Q$ as soon as $v(Q) = 1$. Therefore neither consistency nor regularity with respect to negation are a defining property of paraconsistent semantics. Instead a model is required to be complete and not trivial.

\begin{prop}
    The following properties characterise the implication.
    \begin{description}[
            labelindent=\parindent,
            before={
                    \renewcommand\makelabel[1]{(##1).}
                }
        ]
        \item[transferrable] If $v(P \then Q) = 1$ then if $v(P) = 1$ then $v(Q) = 1$
        \item[decidable] $v(P \then Q) = 1$ if $v(Q) = 1$ if $v(P) = 1$
    \end{description}
    \begin{proof}
        By modus ponens transferribility holds for any deductively closed set.

        Suppose now $v(Q) = 1$ then $v(P \then Q) = 1$ given materiality of implication. Should $v(P) = 0$ then $v(\neg P) = 1$ by completeness and therefore $v(P \then Q) = 1$ by ex contradictio quod libet.
    \end{proof}
\end{prop}

\begin{dfn}
    A set of propositions $\mathcal E \subseteq \wffs$ is saturable only if there is a model $v$ such that $v(P) = 1$ if $P \in \mathcal E$.

    The set determines the model only if $P \in \mathcal E$ if $v(P) = 1$.

    A model set is deductively closed and maximally not trivial.
\end{dfn}

\begin{lmm}
    A deductively closed set which is maximally not trivial is also complete and decidable.
    \begin{proof}
        Consider a map $v: \wffs \longrightarrow \setsm{0,1}$ such that $v(P) = 0$ only if $P \notin \mathcal E$ where $\mathcal E$ is a model set.

        If $v(P) = 0$ then $\mathcal E \cup \setsm P \infers Q, \neg Q$ by maximal triviality so $v(\neg P) = v(P \then Q) = 1$ by deductive closure.
    \end{proof}
\end{lmm}


Note that a maximally consistent set is maximally not trivial if our logic permits it. In that case decidibility of a deductively closed set already follows from completeness. Evidently the preceding also holds for an extension of paraconsistent semantics with a decidibility clause.

In any case, a model set determines a model.

\begin{lmm}
    A set that does not prove a proposition $\mathcal E \nvdash P$ is a subset of a model set $\mathcal E \subseteq \Gamma$ that does not contain the proposition $P \notin \Gamma$.
    \begin{proof}
        Consider $\Gamma_{i+1} = \Delta_i$ if $P \notin \Delta_i$ and $\Gamma_{i+1} = \Gamma_i$ otherwise, where $\Gamma_0 = \mathcal E$ and $\Delta_i = \partial(\Gamma_i \cup \setsm {\wffs_i})$.
        Evidently $\Gamma = \bigcup_{i \in \mathbb N} \Gamma_i$ is deductively closed, since for any finite $\mathcal D \subset \Gamma$ also $\partial\mathcal D \subset \Gamma_i$ for some $i$.

        We can see that $P \then Q \in \Gamma$ since $\wffs_i = P \then Q$ for some $i$ so lest Peirce's law come to effect $\Gamma_i \cup \setsm{P \then Q} \nvdash P$. Moreover, if $R \notin \Gamma$ then $\Gamma_i \cup \setsm R \infers P$ for some $i$ so $\Gamma \cup \setsm R$ is trivial.

        Alternatively, if $\Gamma_0 = \mathcal E \cup \setsm {\neg P}$ is inconsistent we have $\mathcal E \infers P$ by contradiction. Hence $\Gamma_0$ is consistent. We may take $\Gamma_{i+1} = \Delta_i$ if $\Delta_i$ is consistent. By construction $\Gamma$ is maximally consistent. This proof only works if inconsistency implies triviality. It also relies on the consistency of $P \notin \Gamma$.
    \end{proof}
\end{lmm}

\begin{thm}
    $\mathcal E \valids P$ only if $\mathcal E \infers P$.

    The semantics are characteristic for the inferential system. The converse means that the inferential system is semantically complete.
\end{thm}

\chapter{Predicate logic}

We may extend our semantics with a notion of distinct objects and distinct relations. A relation either holds for certain objects or it does not. If there can be only one object in the relation, it may be considered a property of the object.

\section{Syntax}

\begin{dfn}
    An individual variable $x$ is a distinct reference to any object.

    The set of all variables is $\mathbf x$.
\end{dfn}

Note that the distinction applies to the reference and not to the object. Since a variable refers to any object, it refers to no object exclusively.

In what follows we accumulatively define the notion of a formula. $\Phi$ is the set of all formulas. Henceforth we denote the set of all propositional formulas as $\Phi^0$.

\begin{dfn}
    A predicate $P$ of order $n \in \mathbb N$ applies to an $n$-tuple of variables $x_i \in x^n \subseteq \mathbf x$ to formulate $P(x^n) \in \Phi$.

    A variable $x \in \mathbf x$ is in the scope of the formula $P(x^n)$ at position $i$ if $x_i = x$.

    The set of all predicates is $\mathbf P$.  The set of all predicates of order $n$ is $\mathbf P^n$.
\end{dfn}

\begin{dfn}
    A quantifier $q$ may act on a formula $\varphi \in \Phi$ by fixing every instance of a free variable $x \in \mathbf x$ in the scope of $\varphi$ to formulate $(qx\;\varphi(x)) \in \Phi$.

    A variable is free if it has not been fixed anywhere in the scope of the formula.
\end{dfn}

\begin{quant}[Universal]
    If $\left[\univ x \varphi(x)\right]$ holds, then $[\varphi]$ holds for all objects.
\end{quant}

\begin{dfn}
    A formula is closed if all variables in its scope are fixed. Otherwise it is open.

    We consider a formula well formed only if it is closed. In predicate logic such a formula is known as a sentence.

    The set of all sentences is $\mathcal S \subset \Phi$.

    If $q$ is a quantifier and $qx\;\varphi(x)$ is a sentence, $\varphi$ is known as a first order formula on account of its unique free variable. We denote this as $\varphi = \varphi(x)$.

    The set of all first order formulas is $\Phi^1 \subset \Phi$.
\end{dfn}

A first order formula is clearly not a sentence $\mathcal S^1 \cap \mathcal S = \emptyset$.

\begin{dfn}
    A logical operator $o$ of order $n$ may act on any $n$-tuple of formulas $\varphi_i \in \varphi^n \subseteq \Phi$ to form a new formula $o(\varphi^n) \in \Phi$.
\end{dfn}

This permits all operators of propositional logic to act on formulas. This holds a fortiori for formulas which are well formed.

No sentence is a proposition $\mathcal S \cap \wffs = \emptyset$, and no formula is a propositional formula $\Phi \cap \Phi^0 = \emptyset$. Yet a formula $\varphi \in \Phi$ may substitute uniformly $\pi(\varphi \setminus p)$ for a propositional variable $p$ in a propositional formula $\pi \in \Phi^0$. Hence a propositional formula $\pi \in \Phi^0$ does define an equivalence class over a set of formulas $\varGamma \subset \Phi$ such that every $\gamma \in \varGamma$ is an instance of $\pi$.

We may consider $\Xi^p \subset \Phi$ the set of all formulas that are instances of some axiom $\xi \in \Xi^0$ of propositional logic.

\begin{dfn}[Propositional axioms]
    The set of all propositional axiomatic sentences is $\mathcal X^p = \mathcal S \cap \Xi^p$.

    Note that this set may be accomplished by taking $s, t$ as sentential variables in each propositional axiom, rewriting $\Xi^0$ into what we denote $\Xi^s$.
\end{dfn}

Then $\Xi^p_1 = \Phi^1 \cap \Xi^p$ is the set of all propositional axiomatic first order formulas.

\begin{dfn}[Universal axioms]
    For every $\xi \in \Xi^p_1$ we have $\univ x \xi(x)$ as a universal axiomatic sentence. The set of all such sentences is $\mathcal X^u$.

    Note that this set may be accomplished by rewriting $\Xi^0$ with first order formulaic variables $\varphi, \gamma$ and symbolically quantifying universally over a variate variable $x$ noting $\Xi^u$.
\end{dfn}

We may also substitute any $\varphi, \gamma \in \Phi^1$, any $x \in \mathbf x$ and any $s \in \mathcal S$ into the following axioms.

\begin{axm}[Universal distribution]
    $$\left( \univ x \varphi(x) \then \gamma(x) \right) \then \left(\left( \univ x \varphi(x) \right)  \then \left( \univ x \gamma(x) \right) \right)$$
\end{axm}

\begin{axm}[Universal generalisation]
    $$\left( \univ x s \then \varphi(x) \right) \then \left( s \then \univ x \varphi(x) \right)$$
\end{axm}

If $x$ were somewhere in $s$ then $\univ x s \then \varphi(x)$ would not be well formed since $s$ is closed.

\begin{axm}[Universal existence]
    $$\left( \univ x \varphi(x) \right) \then \left( \neg \univ x \neg\varphi(x) \right)$$
\end{axm}

The set of all sentences which are instances of these axioms is $\mathcal X^d$. We may denote the set of these axioms $\delta$.

\begin{dfn}
    The axioms of first order logic are $\Xi = \Xi^s \cup \Xi^u \cup \delta$. The axiomatic sentences are $\mathcal X = \mathcal X^p \cup \mathcal X^u \cup \mathcal X^d$.
\end{dfn}

\begin{frule}[Modus ponens]
    If $\mathcal M \subseteq \setsm {t \then s, t} \subset (\mathcal M \cup \mathcal X)$ then $\mathcal M \infers s$.
\end{frule}

This assures us of the following. We may denote $\infers_p$ as the rules of propositional inference.

\begin{thm}[Propositional inference]
    Given $\mathcal M, s \subseteq \mathcal S$ an instance of $\Pi, \pi \subseteq \Phi^0$. If $\Pi \infers_p \pi$ then $\mathcal M \infers s$.
    \begin{proof}
        This follows from modus ponens over propositional axioms.
    \end{proof}
\end{thm}

\begin{lmm}[Universal modus ponens]
    $\univ x \varphi(x) \then \gamma(x), \univ x \varphi(x) \infers \univ x \gamma(x)$.
    \begin{proof}
        Likewise over the axiom of universal distribution.
    \end{proof}
\end{lmm}

\begin{thm}[Universal inference]
    Given $\varGamma, \varphi \subseteq \Phi^1$ an instance of $\Pi, \pi \subseteq \Phi^0$ and $\mathcal U_i = \univ x \varGamma_i(x)$ for all $i \leq |\varGamma|$. If $\Pi \infers_p \pi$ then $\mathcal U \infers \univ x \varphi(x)$.
    \begin{proof}
        By universal modus ponens over the universal axioms.
    \end{proof}
\end{thm}

We are ready to consider an alternative quantifier.

\begin{quant}[Existential]
    $$\exis x \varphi(x) \eqv \neg\univ x \neg\varphi(x)$$
    If $\left[ \exis x \varphi(x) \right]$ holds, then there is an object for which $[\varphi]$ holds.
\end{quant}

The following lemma expresses the law of universal existence.

\begin{lmm}
    $\univ x \varphi(x) \infers \exis x \varphi(x)$.
\end{lmm}

The existence of an object satisfying the implicans of a universally quantified implication infers an implied sentence.

\begin{thm}[Existential modus ponens]
    $$\univ x \varphi(x) \then s, \exis x \varphi(x) \infers s$$
    \begin{proof}
        $\univ x \varphi(x) \then s \infers \univ x \neg s \then \neg\varphi(x)$.
    \end{proof}
\end{thm}

Note that $s$ is closed so it may not contain $x$ if $\univ x \varphi(x) \then s$ is to be well formed.

\begin{crl}
    $\univ x \varphi(x) \then s, \univ x \varphi(x) \infers s$.
\end{crl}

Now we quantify universally over both members of the implication. The following shows that we may diminish them to existential clauses.

\begin{lmm}
    $\univ x \varphi(x) \then \gamma(x) \infers \exis x \varphi(x) \then \exis x \gamma(x)$
    \begin{proof}
        $\univ x \varphi(x) \then \gamma(x) \infers \univ x \neg\gamma(x) \then \neg\varphi(x)$.
    \end{proof}
\end{lmm}

The identity of the object satisfying the implicans may be preserved.

\begin{lmm}
    $\univ x \varphi(x) \then \gamma(x) \infers \univ x \varphi(x) \then \varphi(x) \conj \gamma(x)$.
\end{lmm}

\begin{thm}[Individual modus ponens]
    $$\univ x \varphi(x) \then \gamma(x), \exis x \varphi(x) \infers \exis x \varphi(x) \conj \gamma(x)$$
\end{thm}

The classical approach allows us to eliminate a universal quantifier through the introduction of a universal instance.

\begin{dfn}
    An individual constant is a distinct reference to a particular object.

    A constant may not be quantified over, but may substitute uniformly for a free variable in a first order formula to form a sentence.

    The set of all constants is $\mathbf c$.
\end{dfn}

\begin{frule}[Universal instantiation]
    $$\univ x \varphi(x) \infers \varphi(c)$$
\end{frule}

Note that the instance is always universal. There is no such thing as an existential instance. Hence the particularity of the object referred to by the constant is a semantic device and not an identification with an independent object.

It allows us to derive universal existence. We may even show a form which is stronger with respect to the new notation.

\begin{thm}[Existential generalisation]
    $$\varphi(c) \infers \exis x \varphi(x)$$
    \begin{proof}
        $\theorem \neg\varphi(c) \then \neg\univ x \varphi(c)$ and $\univ x \neg\varphi(x) \infers \neg\varphi(c)$.
    \end{proof}
\end{thm}

Note that $x$ may not occur in $\varphi(c)$ if the conclusion is to be well formed.

We may also permit universal generalisation of an instance, under conditions better understood when considering the quantified variety.

\begin{frule}[Universal generalisation] If $c$ does not occur anywhere in $s$:
    $$s \then \varphi(c) \infers \univ x \varphi(x)$$
\end{frule}

Once more $x$ is nowhere in $\varphi(c)$.

This immediately implies the following by familiar reasoning.

\begin{thm}[Existential modus ponens]  If $c$ does not occur anywhere in $s$:
    $$\varphi(c) \then s, \exis x \varphi(x) \infers s$$
\end{thm}

Additionally we may affirm the following.

\begin{thm}[Universal distribution]
    $\;$
    \begin{proof}
        $\univ x \varphi(x) \then \gamma(x), \univ x \varphi(x) \infers (\univ x \varphi(x)) \then \gamma(c)$ where $c$ does not occur in $\univ x \varphi(x)$.
    \end{proof}
\end{thm}

Note that this implies we may omit the universal axioms as well as all additional axioms of first order logic that are not propositional.

\section{Semantics}

\begin{dfn}
    The universe $\mathrm U$ consists of all objects. Even without bound we may not be able to enumerate all objects.
\end{dfn}

\begin{dfn}
    An assignment consists of a map $u$ and a set $p$ such that there is a map $p^n \in p$ for each $n \in \mathbb N$.
    \begin{enumerate}
        \item $u: \mathbf c \longrightarrow \mathrm U$
        \item $p^n: \mathbf P^n \longrightarrow \wp(\mathrm U^n)$
    \end{enumerate}
\end{dfn}

\begin{dfn}
    An assignment $u$, $p$ may be interdeterminate with a valuation map $v: \mathrm P \longrightarrow \setsm{0,1}$ of constant predicative formulas $\mathrm P \subset \Phi$.
    $$v(P^n(x^n)) = 1 \text{\;only if\;} u(x^n) \in p^n(P^n)$$
    A structure is a model $v: \Phi \longrightarrow \setsm{0,1}$ whose predicative formulas $\mathrm P \subset \Phi$ are interdeterminate with an assignment $u$, $p$, where quantifying formulas $\mathord\forall \subset \Phi$ are evaluated by consensus over all adjacent models.
    $$v(\univ x \varphi(x)) = 1 \text{\;only if\;} w(\varphi(c)) = 1$$
    for all constants $c$ and all models $w$ such that $p_w = p$.
\end{dfn}

\begin{dfn}
    Given an assignment $v$ and a variable $x$, consider an assignment $w$ such that $w(P^n) = v(P^n)$ for all $P^n$, $n$, and $w(y) = v(y)$ for all $y \neq x$. We say that the structures $w$, $v$ interdeterminate with the assignments $w$, $v$ respectively are $x$-adjacent to one another.

    The $x$-adjacency $\partial_x v$ of a structure $v$ consists of all structures $x$-adjacent to $v$.
\end{dfn}

\begin{dfn}

    Finally, a model $v$ is a structure $v$ where a formula quantifying over $x$ is evaluated by consensus about the quantified formula between all models in the $x$-adjacency of $v$.
    $$v(\univ x \varphi(x)) = 1 \text{\;only if\;} w(\varphi(x)) = 1 \text{\;for all models\;} w \in \delta_x v$$
\end{dfn}


\newpage



A model $v \in \mathbf M$ consists of assignment maps.
Their structure is interdeterminate with a valuation map $v: \Phi \longrightarrow \setsm{0,1}$ of all formulas.
Inferentiality is taken to mean deductive closure of the positive domain over all instances of some propositional inference, regardless of whether a formula in the inference is closed.

We may extend our alphabet with the predicate symbol $P$. $'$ denotes the order of the predicate. For instance, $\mathbf P^2$ may be denoted $P'$. Enclosing in parentheses permits enumeration $'$. For instance, given $P, Q, R \in \mathbf P^2$ we may write $R$ as $(P')''$.


We may appreciate our notion of a formula from a different light.

In the study of propositional logic, formulas were a metatheoretical device that allowed us to introduce substitutional rules into our theory. As such they were not a feature of the logic itself, but of the metalogic. Admittedly, that metalogic was a predicate logic, and the substitution method was of a semantic nature.

In what follows, we will consider the notion of formulas in predicate logic. We conveniently retain use of the symbol $\Phi$ for the set of all formulas. Once more, formulas are a convenient metatheoretical device

Strictly speaking, many formulas are also ill formed in predicate logic. However, the manner in which they may become well formed is contrary to the propositional analogue.

Consider predicates $P$ of order $1$ and $R$ of order $2$. We may apply them to variables $x, y$ to form $P(x)$ and $R(x,y)$. We may then form $\neg P(x)$ and $\univ x R(x, y)$. These are all formulas yet none are well formed. Only $\univ y \univ x R(x,y)$ is a sentence. So is $\neg(\univ y (\univ x R(x,y)) \then \neg P(y))$. However, $\univ x \univ x R(x, x)$ is not even a formula.


\end{document}


