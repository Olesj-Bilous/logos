\documentclass{amsbook}
\usepackage{mathtools}
\usepackage{amssymb}
\usepackage{enumitem}

\newcommand{\setsm}[1]{\left\{#1\right\}}
\newcommand{\given}{\mathrel |}

\newcommand{\wffs}{\mathcal W}

\newcommand{\infers}{\mathrel\vdash}
\newcommand{\theorem}{\mathord\vdash\medspace}
\newcommand{\univ}[1]{\mathord\forall#1\;}
\newcommand{\exis}[1]{\mathord\exists#1\;}

\newcommand{\valids}{\mathrel\vDash}


\newcommand{\then}{\mathrel\rightarrow}
\newcommand{\conj}{\mathrel\&}
\newcommand{\eqv}{\mathrel\leftrightarrow}
\newcommand{\disj}{\mathrel\vee}

\DeclareMathOperator*{\amp}{\&}

\theoremstyle{definition}
\newenvironment{hackthm}[2][section]{
    \newtheorem{new}{#2}[#1]
    \begin{new}
}{\end{new}}

\newtheorem{axm}{Axiom}[chapter]
\newtheorem{prop}{Property}[section]
\newtheorem{subprop}{Property}[subsection]
\newtheorem{frule}{Rule}[chapter]
\newtheorem{subrule}{Rule}[subsection]
\newtheorem{op}{Operator}[chapter]

\newtheorem{quant}{Quantifier}[chapter]

\newtheorem{thm}{Theorem}[section]
\newtheorem{subthm}{Theorem}[subsection]
\newtheorem{lmm}{Lemma}[section]
\newtheorem{crl}{Corollary}[section]
\newtheorem{dfn}{Definition}[section]


\begin{document}


\title{Metamathema}
\author{Olesj Bilous}
\maketitle

\chapter{Propositional logic}

A proposition $P$ signifies a distinct fact $P^*$ that is either the case or not.

The set of all propositions is $\mathcal W$.

\section{Syntax}

\begin{dfn}
    A logical operator $o$ of order $n \in \mathbb N$ may act on any $n$-tuple of propositions $P^n \in \mathcal W^n$ to compose a proposition $o(P^n) = P \in \mathcal W$.

    Each $P_i \in P^n$ is said to be the $i$th direct descendant of $P$. A proposition $Q$ is said to be a descendant of $P$ if there is a succession of direct descendants that leads from $P$ to $Q$.

    When a proposition $P$ is atomic it is not composed $P \neq o(P^n)$ of any $n$-tuple $P^n \in \wffs^n$ by any operator $o$ of any order $n$.
\end{dfn}

\begin{op}[Implication]
    For any $P, Q \in \wffs$:
    $$
        P \rightarrow Q \in \wffs.
    $$
    If $(P \rightarrow Q)^*$ is the case, then if $P^*$ (the implicans) is the case, then so is $Q^*$ (the implicandum). The implicans is said to be a sufficient condition for the implicandum.
\end{op}

\begin{op}[Negation]
    For any $P \in \wffs$:
    $$
        \neg P \in \wffs.
    $$
    If $(\neg P)^*$ is the case then $P^*$ is not the case.
\end{op}

\begin{dfn}
    Given an alphabet $\alpha$ consisting of a finite $|\alpha| \in \mathbb N$ set of symbols, we consider any finite $n \in \mathbb N$ string of symbols from the alphabet $e^n \in \alpha^n$ an expression. An instance $e_i \in e^n$ of a symbol $e_i = \alpha_j \in \alpha$ at position $i$ is known as a character.

    The set of all expressions from the alphabet is $\alpha^\omega = \bigcup_{n \in \mathbb N} \alpha^n$.

    The substring $d^m = \left[ e^n \right]_{k+1}^{k+m}$ of an expression $e^n \in \alpha^n$ at position $k + 1$ of length $m$ is its restriction to the characters $e_i \in e^n$ at positions $k < i \leq k+m$. The instance $\left[ e^n \right]_{k+1}^{k+m}$ of the expression $d^m$ is said to be in the scope of the expression $e^n$ at position $k+1$.

    Uniform substitution $e(d \setminus c)$ of an expression $c \in \alpha^l$ with an expression $d \in \alpha^m$ in an expression $e \in \alpha^n$ substitutes an instance of $d$ for every instance of $c$ in the scope of $e$.
\end{dfn}

\begin{thm}
    The alphabet $\beta = \setsm{\mathrm a, \mathrm o, ', (, ),\,,}$ expresses all propositions.
    $$ \wffs \subset \beta^\omega $$
    \begin{proof}
        We use the atom symbol $\mathrm a$ and the succession symbol $'$ to denote an atomic proposition for every $n \in \mathbb N$ such that $a_0 = \mathrm a$ and $a_{n+1} = a_n'$.

        We may use the operator symbol $\mathrm o$ and the succession symbol $'$ to denote any operator. Then parentheses $($, $)$ and a comma $,$ allow us to denote components in a composite proposition.
    \end{proof}
\end{thm}

\begin{crl}
    The set of all propositions $\mathcal W$ may be enumerated though without bound: $|\wffs| = |\mathbb N|$.
    \begin{proof}
        The alphabet $\beta$ allows any proposition to be encoded in base $6$.
    \end{proof}
\end{crl}

\begin{crl}
    A proposition is in the scope of another only if it descends from the other.
\end{crl}

\begin{dfn}
    A propositional variable $p$ is a distinct reference to any proposition $P$. An extension $\phi = \beta \cup \setsm{\mathrm p}$ of the alphabet $\beta$ expresses all such variables.

    A formula expresses the logical structure of a proposition. We define the set $\Phi$ of all formulas recursively.

    If $\varphi \in \Phi$ is a formula then so is the result of uniform substitution $\varphi(p \setminus P) \in \Phi$ of any proposition $P$ in the scope of $\varphi$ with any variable $p$. Furthermore, $\wffs \subset \Phi$. Note that $\Phi \subset \phi^\omega$. A formula $\varphi \in \Phi$ is said to be well formed only if it is a proposition $\varphi \in \wffs$.

    A proposition $P$ is said to be an instance of a formula $\varphi$ only if it follows from uniform substitution $P = \varphi(P_i \setminus p_i)^n$ of all propositional variables $p_{i \leq n}$ in the scope of $\varphi$ with some proposition $P_i$.
\end{dfn}

\begin{dfn}
    A set of propositions $\mathcal E \subseteq \wffs$ may allow us to infer $\mathcal E \infers P$ a proposition $P \in \wffs$. The propositions of $\mathcal E$ are known as premises whereas $P$ is said to be a syntactic consequence of $\mathcal E$.

    The deductive closure $\partial \mathcal E$ contains all syntactic consequences of the set such that $\mathcal E \infers P$ only if $P \in \partial\mathcal E$. The set $\mathcal E$ may have the following properties.
    \begin{enumerate}[
            labelindent=\parindent,
            before={
                    \renewcommand\makelabel[1]{(##1).}
                }
        ]
        \item[deductively closed] $\mathcal E = \partial\mathcal E$
        \item[consistent] if $P \in \partial\mathcal E$ then $\neg P \notin \partial\mathcal E$
        \item[trivial] $\partial\mathcal E = \wffs$
    \end{enumerate}

    An extension of the propositional alphabet $\beta^{\mathord\vdash} = \beta \cup \setsm{\mathord\vdash}$ allows the inference symbol $\mathord\vdash$ to act on any finite set of propositions to compose an inferential expression $I \in \wffs^{\mathord\vdash}$.

    A propositional variable may substitute uniformly $\eta = \iota(p \setminus P)$ for a proposition in an inferential formula $\iota, \eta \in \Phi^{\mathord\vdash}$ where $\wffs^{\mathord\vdash} \subset \Phi^{\mathord\vdash}$ and $\wffs^{\mathord\vdash}$ expressible from $\phi^{\mathord\vdash} = \phi \cup \setsm{\mathord\vdash} = \beta^{\mathord\vdash} \cup \setsm{\mathrm p}$.

    A pure inferential formula has no atomic propositions in its scope.

    An inferential expression is an instance of an inferential formula if it follows from uniform substitution of all propositional variables in its scope.

    A rule of inference is a pure inferential formula such that each instance of the formula expresses a valid inference.
\end{dfn}

Note that not every inferential expression is a valid inference, nor every inferential formula a rule of inference. However, postulating certain properties and rules of inference allows us to determine all rules of inference. Foundational properties and rules will be superscribed with a dagger $\dag$.

The first property ensures that inference is structural, and that no set of propositions is privileged.

\begin{prop}[Universality]
    If $\mathcal E \infers P$ then $\mathcal E(Q \setminus a) \infers P(Q \setminus a)$ where $Q$ is any proposition and $a$ any atomic proposition.
\end{prop}

Equivalently, every inference is an instance of some rule of inference, namely by uniform substitution of each atom in the scope of the inference with a variable such that each variable substitutes for only one atom.

\begin{prop}[Transitivity]
    If $\mathcal E \infers P$ for all $P \in \mathcal D$ and $\mathcal D \infers Q$ then $\mathcal E \infers Q$.
\end{prop}

The following property does not hold in adaptive inference systems.

\begin{prop}[Monotony]
    If $\mathcal E \infers P$ then $\mathcal E \cup \setsm Q \infers P$ for any $Q$.
\end{prop}

\begin{frule}[Modus ponens]
    $$\begin{aligned}p \rightarrow q, \\ p\end{aligned} \infers q$$
\end{frule}

\subsection{Inferential approach}

The following property allows us to infer over the possibility of inference itself. Consider the extension $\mathcal E \cup\setsm P$ of some premises $\mathcal E$ with a proposition $P$ known as the hypothesis.

\begin{subprop}[Deduction]
    If $\mathcal E \cup\setsm P \infers Q$ then
    $$\mathcal E \infers P \then Q.$$
    for all $\mathcal E \subset \wffs$.
\end{subprop}

Hence the subject of hypothesis is sufficient condition for what can be inferred from the extension.

This holds a fortiori for any finite $\varLambda \subset \Phi$ such that $\varLambda \cup \setsm p \infers q$ since it holds for all instances.

Furthermore this entails a familiar property.

\begin{thm}[Linearity of inference]
    If $\varLambda \infers p$ and $\varLambda \cup \setsm p \infers q$ then $\varLambda \infers q$ for all $\varLambda \subset \Phi$.
    \begin{proof}
        From $\varLambda \infers p \then q$ and transitivity of inference.
    \end{proof}
\end{thm}

Some expected properties of the implication can now follow.

\begin{thm}[Transitivity of implication]
    $$\begin{aligned}
            p \then q, \\
            q \then r
        \end{aligned} \infers p \then r$$
    \begin{proof}
        From $p \then q, q \then r, p \infers r$.
    \end{proof}
\end{thm}

\begin{thm}[Distributivity of implication]
    $$p \then (q \then r) \infers (p \then q) \then (p \then r)$$
\end{thm}

The following rule seems all too evident.

\begin{subrule}[Reflection]
    $$p \infers p$$
\end{subrule}

It allows us to prove the following theorems.

\begin{thm}[Reflexivity of implication]
    $$\theorem p \then p$$
\end{thm}

\begin{thm}[Ex quod libet datum]
    $$p \infers q \then p$$
    \begin{proof}
        $p, q \infers p$ by monotony.
    \end{proof}
\end{thm}

The implication thus defined is said to be material.

The preceding rules and theorems have applied exclusively to the implication. The following rule introduces the syntactic role of the negation.

\begin{frule}[Contraposition]
    $$p \then q \infers \neg q \then \neg p$$
\end{frule}

This shows a property of inference.

\begin{thm}[Contrapositivity]
    If $p \infers q$ then $\neg q \infers \neg p$.
\end{thm}

\begin{thm}[Ex absurdum negatio quod libet]
    $$p, \neg p \infers \neg q$$
    \begin{proof}
        $p \infers q \then p$.
    \end{proof}
\end{thm}

\begin{crl}[Ex absurdo falsum]
    For any formula $\tau$ such that $\theorem \tau$:
    $$p, \neg p \infers \neg\tau$$
\end{crl}

\begin{thm}[Ex falso negatio quod libet]
    For any formula $\tau$ such that $\theorem \tau$:
    $$\theorem \neg\tau \then \neg q$$
    \begin{proof}
        $\theorem q \then \tau$, also known as ex quod libet verum.
    \end{proof}
\end{thm}

\begin{frule}[Introduction of double negation]
    $$p \infers \neg\neg p$$
\end{frule}

\begin{lmm}[Contraposition with left elimination]
    $$p \then \neg q \infers q \then \neg p$$
    \begin{proof}
        $q \then \neg\neg q$ is transitive with $\neg\neg q \then \neg p$.
    \end{proof}
\end{lmm}

\begin{thm}[Negatio improbis]
    For any formula $\tau$ such that $\theorem \tau$:
    $$p \then \neg \tau \infers \neg p.$$
\end{thm}

\begin{thm}[Introduction of implicative conjuction]
    $$p, q \infers \neg(p \then \neg q)$$
    \begin{proof}
        We can deduce $p \infers (p \then \neg q) \then \neg q$ from modus ponens.
    \end{proof}
\end{thm}

This has a special case with an interesting consequence.

\begin{lmm}[Negatio reprobis]
    $$p \then \neg p \infers \neg p$$
    \begin{proof}
        $p \infers \neg(p \then \neg p)$.
        Alternatively by ex contradictio falsum.
    \end{proof}
\end{lmm}

\begin{thm}[Reductio ad absurdum]
    $$\begin{aligned}
            p \then q, \\ p \then \neg q
        \end{aligned} \infers \neg p$$
\end{thm}

By contrapositivity we also have the following.

\begin{crl}[Commutativity of implicative conjunction]
    $$\neg(p \then \neg q) \infers \neg(q \then \neg p)$$
\end{crl}

The final rule completes the law of double negation.

\begin{frule}[Elimination of double negation]
    $${\neg\neg p} \infers p$$
\end{frule}

This rule has been controversial since it allows the following.

\begin{thm}[Proof by contradiction]
    $$\neg p \then q, \neg p \then \neg q \infers p$$
\end{thm}

This result is rejected by intuitionists, hence they do not allow double negation to be eliminated.

The following completes the implicative variety of the law of junctive commutativity.

\begin{lmm}[Commutativity of implicative disjunction]
    $$\neg p \then q \infers \neg q \then p$$
\end{lmm}

This leads to double negation elimination considering $\theorem \neg p \then \neg\neg\neg p$.

Nonetheless, the following results may be preserved by intuitionists as independent rules. We complete the law of implicative conjuction and show an interesting lemma.

\begin{crl}[Elimination of implicative conjunction]
    $$\neg(p \then \neg q) \infers p, q$$
    \begin{proof}
        $\neg p \infers p \then \neg q$.
    \end{proof}
\end{crl}

\begin{lmm}[Introduction of implicative disjunction]
    $$p \infers \neg p \then q$$
    \begin{proof}
        $p \infers \neg q \then p$.
    \end{proof}
\end{lmm}

\begin{thm}[Ex absurdo quod libet]
    $$\begin{aligned}
            p, \\ \neg p
        \end{aligned}\infers q$$
\end{thm}

A consistent set is clearly not trivial. Now we can also show the following.

\begin{crl}
    An inconsistent set is trivial.
\end{crl}

Hence a maximally consistent set is maximally not trivial. This is avoided in paraconsistent inferential systems.

For any formula $\tau$ such that $\theorem \tau$:

\begin{thm}[Ex falso quod libet]
    $$\theorem \neg\tau \then q$$
\end{thm}

\begin{crl}[Ex falso absurdum]
    $$\neg\tau \infers p, \neg p$$
\end{crl}

Furthermore the completion of the law of contraposition allows us to conclude with a remarkable result.

\begin{lmm}[Converse contraposition]
    $$\neg p \then \neg q \infers q \then p$$
\end{lmm}

\begin{thm}[Peirce's law]
    $$(p \then q) \then p \infers p$$
    \begin{proof}
        We know that $\theorem \neg p \then (\neg q \then \neg p)$ so $\theorem \neg p \then (p \then q)$. However, in that case $\neg p \then \neg(p \then q) \infers p$.
    \end{proof}
\end{thm}

\begin{crl}[Consequentia mirabilis]
    $$\neg p \then p \infers p$$
    \begin{proof}
        $\theorem (p \then \neg p) \then \neg p$ is transitive with $\neg p \then p$.
    \end{proof}
\end{crl}

By ex absurdum quod libet we can show $\neg\neg p, \neg p \infers p$. Evidently Peirce's law is unacceptable to intuitionists.

The following rule is rejected in its implicative variety for similar reasons.

\begin{thm}[Implicative dilemma]
    $$\neg p \then q, \begin{aligned}
            p \then r \\
            q \then r
        \end{aligned} \infers r$$
    \begin{proof}
        $p \then q, \neg p \then q, \neg q \infers q$.
    \end{proof}
\end{thm}

Since $\theorem p \then p$ this is enough to invite consequentia mirabilis.

\subsection{Alternative operators}

We may disregard the law of double negation and introduce some new operators.

\begin{op}[Conjuction]
    If $(P \conj Q)^*$ is the case, then $P^*$ is the case and $Q^*$ is the case.
\end{op}

\begin{frule}[Law of conjunction]
    $p, q \infers p \conj q$ and $p \conj q \infers p, q$.
\end{frule}

\begin{thm}[Conjunctive syllogism]
    $\neg(p \conj q), p \infers \neg q$.
    \begin{proof}
        $p \infers q \then (p \conj q)$.
    \end{proof}
\end{thm}

\begin{frule}[Law against contradiction]
    $$\theorem \neg(p \conj \neg p)$$
\end{frule}

\begin{lmm}
    $\theorem p \then \neg\neg p$.
\end{lmm}

\begin{lmm}
    $p \conj q \infers \neg(p \then \neg q)$
\end{lmm}

\begin{lmm}
    $p \then q \infers \neg(p \conj \neg q)$.
\end{lmm}

\begin{op}[Disjunction]
    If $(P \disj Q)^*$ is the case, then $P^*$ is the case or $Q^*$ is the case.
\end{op}

\begin{frule}[Disjunctive syllogism]
    $$p \disj q, \neg p \infers q$$
\end{frule}

\begin{lmm}
    $\neg p \disj q \infers p \then q$.
\end{lmm}

\begin{frule}[Introduction of disjunction]
    $$p \infers p \disj q$$
\end{frule}

\begin{frule}[Dilemma]
    $$p \disj q,\begin{aligned}
            p \then r, \\ q \then r
        \end{aligned} \infers r$$
\end{frule}

\begin{frule}[Law of junctive commutativity]
    $p \conj q \infers q \conj p$ and $p \disj q \infers q \disj p$.
\end{frule}

\begin{op}[Equivalence]
    If $(P \eqv Q)^*$ is the case then $P^*$ is the case only if $Q^*$ is the case.
\end{op}

\begin{frule}[Law of equivalence]
    \begin{align*}
        p \then q, q \then p & \infers p \eqv q             \\
        p \eqv q             & \infers p \then q, q \then p
    \end{align*}
\end{frule}

\begin{thm}
    $\theorem (p \eqv q) \eqv ((p \then q) \conj (q \then p))$
\end{thm}

\begin{thm}[De Morgan's laws I]
    $$\theorem \neg(p \disj q) \eqv (\neg p \conj \neg q)$$
    \begin{proof}
        $\theorem p \then (p \disj q)$, $\theorem q \then (p \disj q)$. $\neg p \infers (p \disj q) \then q$.
    \end{proof}
\end{thm}

\begin{thm}[De Morgan's laws IIa]
    $$\neg p \disj \neg q \infers \neg(p \conj q)$$
    \begin{proof}
        $\neg p \disj \neg q, p \conj q \infers q, \neg q$.
    \end{proof}
\end{thm}

\begin{frule}[Law of the excluded middle]
    $$\theorem \neg p \disj p$$
\end{frule}

\begin{crl}
    $\theorem \neg\neg p \then p$.
\end{crl}

This is the first offence against intuitionism in this approach.

A weaker variety of the offending law suffices to complete De Morgan's laws.

\begin{lmm}
    $\theorem \neg p \disj \neg\neg p$
\end{lmm}

\begin{thm}[De Morgan's laws IIb]
    $$\neg(p \conj q) \infers \neg p \disj \neg q$$
    \begin{proof}
        $\neg(p \conj q) \infers \neg\neg p \then (\neg p \disj \neg q)$ from $p \then \neg q \infers \neg\neg p \then \neg q$.
    \end{proof}
\end{thm}

\begin{thm}
    \begin{align*}
         & \theorem & (p \conj q) &  & \eqv &  & \neg & (p \then \neg q) & \\
         & \theorem & (p \disj q) &  & \eqv &  &      & (\neg p \then q) &
    \end{align*}
    \begin{proof}
        From elimination of implicative conjuction and from $\neg p \then q \infers \neg p \then (p \disj q)$ by dilemma over exclusion of the middle.
    \end{proof}
\end{thm}

\begin{crl}
    \begin{align*}
         & \theorem & (p \then q) &  & \eqv &  & \neg & (p \conj \neg q) & \\
         & \theorem & (p \then q) &  & \eqv &  &      & (\neg p \disj q) &
    \end{align*}
\end{crl}

\subsection{Axiomatic approach}

We may disregard all properties and rules from the inferential approach. Instead we consider a special set of formulas $\Xi \subset \Phi$ known as axioms. We may also consider the set of all instances of axioms $\mathcal X \subset \wffs$.

\begin{prop}
    If $\mathcal E \subset \setsm {Q, Q \then P} \subset (\mathcal E \cup \mathcal X)$ then $\mathcal E \infers P$.
\end{prop}

This property is independent from standard modus ponens yet extends it. It is also independent from monotony.

\begin{axm}
    $p \then (q \then p)$
\end{axm}

\begin{axm}
    $(p \then (q \then r)) \then ((p \then q) \then (p \then r))$
\end{axm}

\begin{lmm}
    $\theorem p \then p$
    \begin{proof}
        Taking $\alpha = p \then ((q \then p) \then p)$ and $\beta = p \then (q \then p)$ we have $\alpha \then (\beta \then (p \then p)), \alpha, \beta \in \Xi$.
    \end{proof}
\end{lmm}

\begin{crl}
    $p \infers p$
\end{crl}

\begin{thm}[Deduction]
    If $\varLambda \cup \setsm p \infers q$ then $\varLambda \infers p \then q$.
    \begin{proof}
        We have $\varLambda \infers p \then \alpha$ for any $\alpha \in \varLambda \cup \setsm p$.

        Hence for any $\alpha \then \beta \in \varLambda \cup \setsm p$ we have $\varLambda \infers p \then (\alpha \then \beta)$.

        Evidently $\varLambda \infers p \then \beta$ permits continued deduction until we show $\varLambda \infers p \then \gamma$ for any $\gamma \in \partial(\varLambda \cup \setsm p)$.
    \end{proof}
\end{thm}

\begin{axm}
    $(\neg p \then \neg q) \then (q \then p)$
\end{axm}

\begin{thm}[Double negation elimination]
    $\theorem \neg\neg p \then p$
    \begin{proof}
        $\neg\neg p \infers \neg\neg(\neg\neg p) \then \neg\neg p$.
    \end{proof}
\end{thm}

\begin{thm}[Double negation introduction]
    $\theorem p \then \neg\neg p$
    \begin{proof}
        $\theorem \neg\neg(\neg p) \then \neg p$.
    \end{proof}
\end{thm}

Evidently the axiomatic approach allows precisely the same syntactic derivations as the inferential.


\section{Semantics}

The expressive range of $P$ is restricted to a binary distinction $P^* \in \setsm{0,1}$.

We say that $P^*$ is the case only if $P^* = 1$. Only in that case is $P$ true.

\begin{dfn}
    A model is a valuation map $v: \wffs \longrightarrow \setsm{0,1}$ that assigns a truth value $v(P) \in \setsm{0,1}$ to each proposition $P \in \wffs$.

    Support for $(\mathcal E, P)$ by a model $v$ means that if all $E \in \mathcal E$ assign to $v(E) = 1$, then $v(P) = 1$. Furthermore any $v$ is:
    \begin{description}[
            labelindent=\parindent,
            before={
                    \renewcommand\makelabel[1]{(##1).}
                }
        ]
        \item [inferential] If $\mathcal E \infers P$ then $v$ supports $(\mathcal E, P)$.
        \item [regular with respect to negation] If $v(P) = v(Q)$ then $v(\neg P) = v(\neg Q)$.
        \item [not trivial] There is at least one proposition $P$ such that $v(P) = 0$.
    \end{description}
    We may consider the set of all models $\mathbf M$.
\end{dfn}

The positive domain $v^{-1}(1) \subseteq \wffs$ of a valuation map $v$ contains all propositions $P$ for which $v(P) = 1$. Clearly the model is determined by the positive domain alone. A model $v$ supports $(\mathcal E, P)$ only if $P \in v^{-1}(1)$ if $\mathcal E \subseteq v^{-1}(1)$. We can see that a valuation map is inferential only if the positive domain is deductively closed.

\begin{dfn}
    A proposition is a semantic consequence $\mathcal E \valids P$ of a set of premises only if all models $v \in \mathbf M$ support $(\mathcal E, P)$.

    $\mathcal E$ is said to validate $P$.
\end{dfn}

\begin{lmm}
    If $\mathcal E \infers P$ then $\mathcal E \valids P$.

    The inferential system is semantically sound. The semantics are adequate for the inferential system.
\end{lmm}

\begin{prop}
    The following properties characterise the negation.
    \begin{description}[
            labelindent=\parindent,
            before={
                    \renewcommand\makelabel[1]{(##1).}
                }
        ]
        \item[consistency] $v(\neg P) = 0$ if $v(P) = 1$
        \item[completeness] $v(\neg P) = 1$ if $v(P) = 0$
    \end{description}

    \begin{proof}
        A model must be consistent since it may not be trivial.

        Consider now that $\theorem T$ where $T = a \then \neg\neg a$. Clearly $v(T) = 1$ so $v(\neg T) = 0$ by consistency. However, $\theorem \neg\neg T$ from which $v(\neg P) = 1$ if $v(P) = 0$ by negational regularity.
    \end{proof}
\end{prop}

A consistent map $v$ is not trivial. If it is also complete, it is regular with respect to negation. Hence, an inferential valuation map is regular with respect to negation and not trivial only if it is consistent and complete.

From paraconsistent inferential systems we may derive inferential maps which are not trivial yet inconsistent. Moreover, regularity with respect to negation implies that a single contradiction $v(P) = 1 = v(\neg P)$ of some $P$ supports the contradiction $v(\neg Q) = 1$ of each $Q$ as soon as $v(Q) = 1$. Therefore neither consistency nor regularity with respect to negation are a defining property of paraconsistent semantics. Instead a model is required to be complete and not trivial.

\begin{prop}
    The following properties characterise the implication.
    \begin{description}[
            labelindent=\parindent,
            before={
                    \renewcommand\makelabel[1]{(##1).}
                }
        ]
        \item[transferrable] If $v(P \then Q) = 1$ then if $v(P) = 1$ then $v(Q) = 1$
        \item[decidable] $v(P \then Q) = 1$ if $v(Q) = 1$ if $v(P) = 1$
    \end{description}
    \begin{proof}
        By modus ponens transferribility holds for any deductively closed set.

        Suppose now $v(Q) = 1$ then $v(P \then Q) = 1$ given materiality of implication. Should $v(P) = 0$ then $v(\neg P) = 1$ by completeness and therefore $v(P \then Q) = 1$ by ex contradictio quod libet.
    \end{proof}
\end{prop}

\begin{dfn}
    A set of propositions $\mathcal E \subseteq \wffs$ is saturable only if there is a model $v$ such that $v(P) = 1$ if $P \in \mathcal E$.

    The set determines the model only if $P \in \mathcal E$ if $v(P) = 1$.

    A model set is deductively closed and maximally not trivial.
\end{dfn}

\begin{lmm}
    A deductively closed set which is maximally not trivial is also complete and decidable.
    \begin{proof}
        Consider a map $v: \wffs \longrightarrow \setsm{0,1}$ such that $v(P) = 0$ only if $P \notin \mathcal E$ where $\mathcal E$ is a model set.

        If $v(P) = 0$ then $\mathcal E \cup \setsm P \infers Q, \neg Q$ by maximal triviality so $v(\neg P) = v(P \then Q) = 1$ by deductive closure.
    \end{proof}
\end{lmm}


Note that a maximally consistent set is maximally not trivial if our logic permits it. In that case decidibility of a deductively closed set already follows from completeness. Evidently the preceding also holds for an extension of paraconsistent semantics with a decidibility clause.

In any case, a model set determines a model.

\begin{lmm}
    A set that does not prove a proposition $\mathcal E \nvdash P$ is a subset of a model set $\mathcal E \subseteq \Gamma$ that does not contain the proposition $P \notin \Gamma$.
    \begin{proof}
        Consider $\Gamma_{i+1} = \Delta_i$ if $P \notin \Delta_i$ and $\Gamma_{i+1} = \Gamma_i$ otherwise, where $\Gamma_0 = \mathcal E$ and $\Delta_i = \partial(\Gamma_i \cup \setsm {\wffs_i})$.
        Evidently $\Gamma = \bigcup_{i \in \mathbb N} \Gamma_i$ is deductively closed, since for any finite $\mathcal D \subset \Gamma$ also $\partial\mathcal D \subset \Gamma_i$ for some $i$.

        We can see that $P \then Q \in \Gamma$ since $\wffs_i = P \then Q$ for some $i$ so lest Peirce's law come to effect $\Gamma_i \cup \setsm{P \then Q} \nvdash P$. Moreover, if $R \notin \Gamma$ then $\Gamma_i \cup \setsm R \infers P$ for some $i$ so $\Gamma \cup \setsm R$ is trivial.

        Alternatively, if $\Gamma_0 = \mathcal E \cup \setsm {\neg P}$ is inconsistent we have $\mathcal E \infers P$ by contradiction. Hence $\Gamma_0$ is consistent. We may take $\Gamma_{i+1} = \Delta_i$ if $\Delta_i$ is consistent. By construction $\Gamma$ is maximally consistent. This proof only works if inconsistency implies triviality. It also relies on the consistency of $P \notin \Gamma$.
    \end{proof}
\end{lmm}

\begin{thm}
    $\mathcal E \valids P$ only if $\mathcal E \infers P$.

    The semantics are characteristic for the inferential system. The converse means that the inferential system is semantically complete.
\end{thm}

\chapter{Predicate logic}

We may extend our semantics with a notion of distinct objects and distinct relations. A relation either holds for certain objects or it does not. If there can be only one object in the relation, it may be considered a property of the object.

\section{Syntax}

In predicate logic a well formed formula is known as a sentence.

The set of all sentences is $\mathcal S$.

\begin{dfn}
    An individual constant is a distinct reference to a particular object.

    The set of all constants is $\mathbf c$.
\end{dfn}

\begin{dfn}
    A predicate $P$ of order $n \in \mathbb N$ applies to an $n$-tuple of constants $c_i \in c^n \subseteq \mathbf c$ to formulate a sentence $P(c^n) \in \mathcal S$.

    A constant $c \in \mathbf c$ is in the scope of the sentence $P(c^n)$ at position $i$ if $c_i = c$.

    The set of all predicates is $\mathbf P$. The set of all predicates of order $n$ is $\mathbf P^n \subset \mathbf P$.
\end{dfn}

\begin{dfn}
    An individual variable $x$ is a distinct reference to any object.

    The set of all variables is $\mathbf x$.
\end{dfn}

Note that the distinction applies to the reference and not to the object. Since a variable refers to any object, it refers to no object exclusively.

\begin{dfn}
    A quantifier $q$ may act on a sentence $s \in \mathcal S$ by uniform substitution $\varphi(x) = s(x \setminus c)$ of a constant $c \in \mathbf c$ in the scope of $s = \varphi(c)$ with a variable $x \in \mathbf x$ nowhere in the scope of $s$ to formulate $qx\;\varphi(x) \in \mathcal S$.

    We call $\varphi(x)$ a formula and $x$ a free variable. A formula $\varphi(x^n)$ involved in $0 < n$ levels of quantification has $n$ distinct free variables $x_i \in x^n$ in its scope. Such a formula is considered open. Only a closed formula is well formed.

    A sentence $s \in \mathcal S$ may be an instance $s = \varphi(c^n)$ of an open formula $\varphi(x^n)$ if it follows from uniform substitution $s = \varphi(x^n)(c_i \setminus x_i)^n$ of the free variables $x^n$. The instance is strict if all $c_i$ are distinct and nowhere in the scope of $\varphi(x^n)$.
\end{dfn}

Note that no such distinct $c_i$ is in the scope of $\varphi(x^n)$ only if $\varphi(x^n) = s(x_i \setminus c_i)^n$ with no $x_i$ anywhere in $s$.

\begin{dfn}
    A logical operator $o$ of order $n$ may act on any $n$-tuple of sentences $s_i \in s^n \subset \mathcal S$ to form a new sentence $o(s^n) \in \mathcal S$.
\end{dfn}

This permits all propositional operators to act on sentences.

A sentence $s \in \mathcal S$ may substitute uniformly $\pi(s \setminus p)$ for a propositional variable $p$ in a propositional formula $\pi$. Hence a propositional formula $\pi$ defines an equivalence class on a set of sentences $\mathcal M \subseteq \mathcal S$ such that every $s \in \mathcal M$ is an instance of $\pi$.

We permit all propositional rules of inference to apply to sentences.

\begin{quant}[Universal]
    If $\left[\univ x \varphi(x)\right]$ holds, then $[\varphi]$ holds for all objects.
\end{quant}

\begin{frule}[Universal instantiation]
    $$\univ x \varphi(x) \infers \varphi(x)(c \setminus x)$$
\end{frule}

We introduce the following quantifier by equivalence.

\begin{quant}[Existential]
    If $\left[\exis x \varphi(x)\right]$ holds, then there is an object for which $[\varphi]$ holds.
    $$\theorem \exis x \varphi(x) \eqv \neg\univ x \neg\varphi(x)$$
\end{quant}

\begin{thm}[Existential generalisation]
    $$\varphi(x)(c \setminus x) \infers \exis x \varphi(x)$$
    \begin{proof}
        $\univ x \neg\varphi(x) \infers \neg\varphi(x)(c \setminus x)$.
    \end{proof}
\end{thm}

The condition to the weaker universal form forbids that we assume anything about $c$.

\begin{prop}[Universal generalisation]
    Given a constant $c \in \mathbf c$ nowhere in the scope of a sentence $s \in \mathcal S$.
    $$\text{If\;} s \infers \varphi(c) \text{\;then\;} s \infers \univ x \varphi(c)(x \setminus c)\text.$$
\end{prop}

Note that $\mathcal M \infers \varphi(c)$ does not imply $\mathcal M \infers \univ x \varphi(c)(x \setminus c)$. This does not preclude the following.

\begin{lmm}
    Given finite premises $\mathcal M \subset \mathcal S$ such that $c$ is nowhere in $\mathcal M$.
    $$\text{If\;} \mathcal M \infers \varphi(c) \text{\;then\;} \mathcal M \infers \univ x \varphi(c)(x \setminus c)\text.$$
    \begin{proof}
        Consider $s = \amp_{i=1}^{|\mathcal M|} \mathcal M_i$.
    \end{proof}
\end{lmm}

Furthermore we may contrapose into the following.

\begin{thm}[Existential instantiation]
    Given $c$ nowhere in $s$ nor in $\mathcal M$ assumed finite.
    $$\text{If\;} \mathcal M \cup \setsm{\varphi(c)} \infers s \text{\;then\;} \mathcal M \cup \setsm{\exis x \varphi(c)(x \setminus c)} \infers s$$
    \begin{proof}
        Clearly $\mathcal M \cup \setsm{\neg s} \infers \neg\varphi(c)$.
    \end{proof}
\end{thm}

This is clearly a weaker form of instantiation which avoids inferring an actual instance. This is necessary to avoid $\exis x \varphi(x) \infers \univ x \varphi(x)$.

\section{Semantics}

\begin{dfn}
    The universe $\mathrm U$ consists of all objects. Even without bound we may not be able to enumerate all objects.
\end{dfn}

\begin{dfn}
    An assignment consists of a map $u$ and a set $r$ such that there is a map $r^n \in r$ for each $n \in \mathbb N$.
    \begin{enumerate}
        \item $u: \mathbf c \longrightarrow \mathrm U$.
              Each constant $c$ is assigned an object $u(c)$.
        \item $r^n: \mathbf P^n \longrightarrow \wp(\mathrm U^n)$.
              Each predicate $P$ of order $n$ is assigned a set $r^n(P)$ of $n$-tuples of objects.
    \end{enumerate}
    We may note $u(c^n)$ where $u(c^n)_i = u(c_i)$ for all $c_i \in c^n \subset \mathbf c$, and $r(P) = r^n(P)$ for $P \in \mathbf P^n$.
\end{dfn}

\begin{dfn}
    A structure $v, (u, r)$ consists of a model $v: \Phi \longrightarrow \setsm{0,1}$ and an assignment $u$, $r$.
    \begin{enumerate}
        \item
              The valuation of constant predicative formulas $\mathrm P \subset \Phi$ is interdeterminate with the assignment.
              $$v(P(c^n)) = 1 \text{\;only if\;} u(c^n) \in r(P)$$
              We say that the $n$-tuple of objects $u(c^n)$ satisfies the formula $P$.
        \item
              Quantifying formulas $\mathord\forall \subset \Phi$ are evaluated by universal consensus on the quantified formula between all structures $w, (t, r)$ interdeterminate with the same predicative assignment $r$.
              $$v(\univ x \varphi(x)) = 1 \text{\;only if\;} w(\varphi(c)) = 1 \text{\;for\;} \begin{aligned}
                       & \text{all constants\;} c \\
                       &
                      \text{all structures\;}
                      w, (t, r)
                  \end{aligned}$$
    \end{enumerate}
\end{dfn}

\begin{dfn}
    A sentence is a semantic consequence of a set of premises $\mathcal M \valids s$ only if all structures $v, (u, r)$ have a model $v$ that supports $(\mathcal M, s)$.
\end{dfn}

\begin{dfn}
    We denote the set of formulas with instances of precisely $n$ distinct constants as $\Phi^n$. Note that $\Phi = \bigcup_{n \in \mathbb N} \Phi^n$.

    We define $\mathrm U^0 = \setsm\emptyset$ such that $\wp(\mathrm U^0) = \setsm{\emptyset, \setsm\emptyset}$. Note that $\emptyset \times S = \emptyset$ and $\setsm\emptyset \times S = S$ for any set $S$.


    A model $v$ may be interdeterminate with a total assignment $u, R$ such that $R^n \in R$ for each $n \in \mathbb N$ where $R^n: \Phi^n \longrightarrow \wp(\mathrm U^n)$.
    $$v(\varphi(c^n)) = 1 \text{\;only if\;} u(c^n) \in R(\varphi)$$
    \begin{enumerate}
        \item $R^n(\neg\varphi) = R^n(\varphi)^C$ where the complement is over $\mathrm U^n$.
        \item $R^{m+n-k}(\varphi \then \gamma) = \setsm{u \in \mathrm U^{m+n-k} \mathrel\bigg| \begin{aligned}
                           & \left[ u \right]_1^m \in R^m(\neg\varphi)        \\
                          \text{\;or\;}
                           & \left[ u \right]_{m-k+1}^{m-k+n} \in R^n(\gamma)
                      \end{aligned} }$ \\
              where common constants are arranged from $m-k+1$ to $m$ if any.

              We define $R^0(\univ x \varphi(x)) = \setsm\emptyset$.
        \item If $v(\univ x \varphi(x)) = 1$ then $R^{n+1}(\varphi(c)) = R^n(\univ x \varphi(x)) \times \mathrm U$.
        \item If $v(\univ x \varphi(x)) = 0$ then $R^{n+1}(\varphi(c)) \subset R^n(\univ x \varphi(x)) \times \mathrm U$.
    \end{enumerate}
\end{dfn}

\begin{lmm}
    A model determines a nonempty set of structures of which it is the model.
    \begin{proof}
        Given any $v$ an injective $u$ ensures that any formula in every $\Phi^n$ has $n$ distinctly assigned constants. This determines a total assignment $R$ and thus a predicative assignment $r \subset R$. We show that $v, (u, r)$ is a structure by inductively decomposing and reconstructing between $R$ and $r$ where $v$ is ever the interdeterminate model.
        \begin{enumerate}
            \item $v(\neg\varphi) = 1$ only if $v(\varphi) = 0$ only if $u(c^n) \notin R(\varphi)$
            \item Similarly from implicational completeness.
            \item Evidently all structures $w, (t, r)$ come to universal consensus on a formula $w(\varphi(c)) = 1$ for all constants $c$ only if all objects satisfy it $R(\varphi) = \mathrm U$.
            \item There must be some object that does not satisfy the formula $q \notin R(\varphi)$ so there must be some structure $w, (t, r)$ assigning some constant $t(c) = q$ such that $w(\varphi(c)) = 0$. Conversely $t(c) \notin R(\varphi)$ if $w(\varphi(c)) = 0$.

                  Note that we may nonetheless have $u(c) \in R(\varphi)$ hence $v(\varphi(c)) = 1$ for any $c$.
        \end{enumerate}
    \end{proof}
\end{lmm}

\begin{thm}[G\"odel]
    First order logic is semantically complete.
\end{thm}

\begin{dfn}

\end{dfn}




\newpage


\end{document}


