Beschouw nu een eindige verzameling $\varGamma \subset \mathbb{N}$ en een bijectie $f: \varGamma \longleftrightarrow \vecrow$ waar $\vecrow$ een basis is voor de vectorruimte $\vecspace$.

Weze $\Phi = \{ \phi \subset \mathbb{N}\times\realnums \ |\ \phi: \varGamma \longrightarrow \realnums \}$.
Men kan de coördinaten van elke vector $\vvec\in\vecspace$ als een functie $\phi\in\Phi$ beschouwen waar $\phi(x)$ de coördinaat is van $\vvec$ naar $f(x)$.
Anderzijds bestaat voor elke vector $\vvec$ een functie $\phi$ volgens zijn coördinaten naar $\vecrow$.
Dus definieert $f$ een bijectie $F: \Phi \longleftrightarrow \vecspace$ tussen reële functies van $\varGamma$ en de vectoren $\vvec\in\vecspace$.

De selectiefuncties $s_x: \varGamma \longrightarrow \setsm{0, 1}$ waar $s_x(t) = 1$ enkel als $t = x$ vormen de deelbijectie met de basis waar $F(s_x) = f(x)$ voor elke $x \in \varGamma$.
Evenwel beschouwt men dit doorgaans vanuit de injectie $i: \varGamma \longrightarrow U(\vecrow)$ .
Hier vervult $f^{-1}$ deze rol, waarbij $U(\vecrow) = \varGamma$.

Nu zouden wij door functies $\phi, \gamma \in \Phi$ wensen te navigeren mits schaal en som, waarbij we $r\phi + \gamma = \eta$ voor $r \in \realnums$ definiëren als $\eta(x) = r\phi(x) + \gamma(x)$.
Evenwel navigeren wij daarbij slechts door vectoren $\vvec\in\vecspace$ middels de overeenkomstige operaties op de coördinaten ten aanzien van $\vecrow$.

Wensen wij nu een andere bijectie te definiëren $g: \varGamma \longleftrightarrow \vecrow[b]$ waar $\vecrow[b] = \vecrow B$ een willekeurige basis is van $\vecspace$.
Dan definieert deze $g$ ook een bijectie $G: \Phi\longleftrightarrow\vecspace$ waar $G(s_x) = g(x) = \vecrow B_i$ voor een zekere $i$.
Daarbij geldt evenzeer $G(\phi)^{\vecrow[b]} = F(\phi)^{\vecrow}$ zodat $G(\phi) = \vecrow BF(\phi)^{\vecrow}$ en dus $G(\phi)^{\vecrow} = BF(\phi)^{\vecrow}$.
Zo transformeren de vectorrepresentaties van elke functie $\phi\in\Phi$ lineair en covariant met de basistransformatie geïmpliceerd door de overgang van $f$ naar $g$.

Beperk $\vecrow[b]$ tot een vrij deel, vervolgens naar believen uitgebreid met lineair afhankelijke vectoren, schrijvend $\vecrow[b]' = \vecrow B'$.
Dan kunnen we een functie beschouwen $g': \varLambda \longrightarrow \vecrow[b]'$.
Voor een willekeurige $U: \vecspace^{\mathbf\omega} \longrightarrow \mathbb{N}^{\mathbf\omega}$ die een zekere injectie definieert $i: \varGamma \longrightarrow U(\vecrow)$ geldt $U(\vecrow[b]') = U(\vecrow B')$ zodat voor $i': \varLambda\longrightarrow U(\vecrow[b]')$ ook geldt $i': \varLambda\longrightarrow U(\vecrow B')$.

Daar een willekeurige basis een lineair gelijkwaardige representatie van $\Phi$ levert noemt men $\vecspace$ een vrije ruimte over $\varGamma$.
Dit is evenwel slechts een concreet geval van een algemener begrip waarvan de definitie buiten het huidig bestek valt.
Daarbij wordt de transformatie van de ruimte volledig bepaald door die van de basis, zoals hier in het geval van vectorruimtes nogmaals mag blijken.